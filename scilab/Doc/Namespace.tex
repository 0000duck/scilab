\newcommand{\Global}{``GLOBAL''}
\newcommand{\env}{``ENVIRONNEMENT''}


Tout d'abord nous introduisons ici la notion de ``scope''.

Un scope d�finit la port�e d'un symbole (que ce soit le nom d'une variable ou d'une fonction).
Si l'on repr�sente les symboles sous forme d'un
``liveness graph'' (graphe de dur�e de vie des noms), un scope repr�sente
un sous-graphe du graphe de plus haut niveau, i.e le programme.

Les espaces de noms dans Scilab regroupent les variables et les functions.
Il est donc impossible d'avoir deux fonctions qui portent le meme nom au sein d'un m�me
scope.

%%
%%	-*- Old Scilab -*-
%%
\section{Scilab : versions ant\'erieures \`a 5.0}

Il existe deux espaces de noms distincts en Scilab :
\begin{itemize}
\item Un espace \Global
\item Un espace ``LOCAL'' que nous appelerons par la suite \env.
\end{itemize}


\subsection{L'espace \Global}

On peut declarer des variables dans cet espace grace au mot clef ``global''.
\begin{verbatim}
global foo
\end{verbatim}
Cette commande permet de d�clarer une variable ``foo'' dans l'espace \Global .

Le comportement de l'espace \Global est assimilable � un tas :
Il ne peut exister qu'un seul et unique symbole representant une variable.


\subsection{L'espace \env}
Le comportement de l'espace \env est assimilable � une pile :
Les variables et les fonctions sont empillables et red�finissables.





%%
%%	-*- New Scilab -*-
%%
\section{Scilab : l'avenir est en marche}
