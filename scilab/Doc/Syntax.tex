Dans cette partie nous decrivons les sp�cificit�s syntaxiques
du language Scilab.


\section{Le token ``ID'' : Les identifiants}
Dans le language Scilab, ne sont consid�r�s
comme des identifiants que les ``mots''
correspondant � l'expression r�guli�re suivante :
\begin{verbatim}
[a-zA-Z_%#][a-zA-Z_0-9#]
\end{verbatim}

\section{Le token ``QUOTE'' : '}
L'utilisation des quotes dans Scilab est assez particuli�re
puisqu'elle prend en charge � la fois les chaines de caracteres et
des operations math�matiques : les transpositions de matrices.

\subsection{Les transpositions matricielles}
Le symbole ``QUOTE'' peut etre utilise pour repr�senter
les transpositions matricielles.

Si $A$ est une matrice, et que nous voulons calculer $^tA$,
ceci s'�crit en language Scilab $A'$.

Nous distinguons donc plusieurs cas ou le comportement
du symbole ``QUOTE'' code une transposition.

\begin{itemize}
\item{Les d�clarations explicites}
$[1, 2, 3 ; 4, 5, 6 ; 7, 8, 9]'$
Lorsque le symbole ``QUOTE'' est pr�c�d� par le token ``RBRACK'',
il code pour une transposition.
\item{Les variables nomm�es}
$a'$
Lorsque le symbole ``QUOTE'' est pr�c�d� par le token ``ID'',
il code pour une transposition.
\item{Les appels de fonctions}
$foo()'$
Lorsque le symbole ``QUOTE'' est pr�c�d� par le token ``RPAREN'',
il code pour une transposition.
\end{itemize}

\subsection{Les chaines de caract�res}
Le symbole ``QUOTE'' peut etre utilise pour repr�senter
des chaines de caract�res.

Nous proposons une impl�mentation des chaines de caract�res de facon homog�ne.
C'est � dire que le ``token'' servant � l'ouverture de la chaine de caract�re
sert aussi a sa fermeture.
\begin{itemize}
\item{Encadr�es par des simples quotes : '}
Les chaines du type :
\begin{verbatim}
'foo' 'bar51'
\end{verbatim}
\item{Encadr�es par des doubles quotes : ``}
Les chaines du type :
\begin{verbatim}
``foo'' ``bar51''
\end{verbatim}
\end{itemize}


\section{Le token CONTINUE : ..}
En fait il ne s'agit pas vraiment d'un token mais d'une r�gle au sein
du lexer permettant l'ecriture de code Scilab sur plusieurs lignes.

Il est reconnaissable via l'expression r�guli�re suivante :
\begin{verbatim}
``..''[\.]*[ \t\v\f]*[\n]
\end{verbatim}

L'utilisation de cette syntaxe n'est pas autoris�e a l'int�rieur de mots clefs.
Exemple :
\begin{verbatim}
whi..
le
\end{verbatim}
Cette syntaxe sera rejet�e au parsing.

Par contre nous autorisons les d�coupes suivantes :
\begin{verbatim}
maVa..
riable
\end{verbatim}
donnera
\begin{verbatim}
maVariable
\end{verbatim}
et
\begin{verbatim}
12345...
6789
\end{verbatim}
donnera
\begin{verbatim}
123456789
\end{verbatim}

Il faudra n�anmoins porter une attention toute particuli�re aux symboles comportants
des points.
\begin{verbatim}
1./...
23456789
\end{verbatim}
vaut
\begin{verbatim}
1./23456789
\end{verbatim}
et non
\begin{verbatim}
1./.23456789
\end{verbatim}

