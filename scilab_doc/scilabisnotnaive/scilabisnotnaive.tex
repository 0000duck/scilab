%
% scilabisnotnaive.tex --
%   Some notes about floating point issues in Scilab.
%
% Copyright 2008 Michael Baudin
%
\documentclass[12pt]{article}

%% Good fonts for PDF
\usepackage[cyr]{aeguill}

%% Package for page headers
\usepackage{fancyhdr}

%% Package to include graphics
%% Comment for DVI
\usepackage[pdftex]{graphicx}

%% Figures formats: jpeg or pdf
%% Comment for DVI
\DeclareGraphicsExtensions{.jpg,.pdf}

%% Package to create Hyperdocuments
%% Comment for DVI
\usepackage[pdftex,colorlinks=true,linkcolor=blue,citecolor=blue,urlcolor=blue]{hyperref}

%% Package to control printed area size
\usepackage{anysize}
%% ...by defining margins {left}{right}{top}{bottom}
\marginsize{22mm}{14mm}{12mm}{25mm}

%% Package used to include a bibliography
\usepackage{natbib}

%% R for real numbers
\usepackage{amssymb}

%% User defined commands

%% Figure reference
\newcommand{\figref}[1]{figure~\ref{#1}}

%% Equation reference
\newcommand{\Ref}[1]{(\ref{#1})}

%% Emphasize a word or a group of words
\newcommand{\empha}[1]{\textit{\textbf{#1}}}

%% Derivation operators
\newcommand{\D}{\partial}
\newcommand{\Dt}{\partial_t}
\newcommand{\Dx}{\partial_x}
\newcommand{\Dy}{\partial_y}

\usepackage{url}

% Scilab macros
\newcommand{\scimacro}[1]{\textit{#1}}
\newcommand{\scicommand}[1]{\textit{#1}}

% To highlight source code
\usepackage{listings}
\usepackage{algorithmic}

% Define theorem environments 
\newtheorem{theorem}{Theorem}[section]
\newtheorem{lemma}[theorem]{Lemma}
\newtheorem{proposition}[theorem]{Proposition}
\newtheorem{corollary}[theorem]{Corollary}

\newenvironment{proof}[1][Proof]{\begin{trivlist}
\item[\hskip \labelsep {\bfseries #1}]}{\end{trivlist}}
\newenvironment{definition}[1][Definition]{\begin{trivlist}
\item[\hskip \labelsep {\bfseries #1}]}{\end{trivlist}}
\newenvironment{example}[1][Example]{\begin{trivlist}
\item[\hskip \labelsep {\bfseries #1}]}{\end{trivlist}}
\newenvironment{remark}[1][Remark]{\begin{trivlist}
\item[\hskip \labelsep {\bfseries #1}]}{\end{trivlist}}

\newcommand{\qed}{\nobreak \ifvmode \relax \else
      \ifdim\lastskip<1.5em \hskip-\lastskip
      \hskip1.5em plus0em minus0.5em \fi \nobreak
      \vrule height0.75em width0.5em depth0.25em\fi}

% Maths shortcuts 
\newcommand{\RR}{\mathbb{R}}

% Algorithms
\usepackage{algorithm2e}
%%\usepackage{algorithmic}

\begin{document}
\author{Michael Baudin}
\date{February 2009}
\title{Scilab is not naive}
\maketitle
\begin{abstract}
Most of the time, the mathematical formula is 
directly used in the Scilab source code. 
But, in many algorithms, some additionnal work is 
performed, which takes into account the fact that 
the computer does not process mathematical real values,
but performs computations with their floating 
point representation.
The goal of this article is to show that, in many
situations, Scilab is not naive and use algorithms
which have been specifically tailored for floating point 
computers. We analyse in this article the 
particular case of the quadratic equation, the 
complex division and the numerical derivatives. 
In each example, we show that the naive algorithm 
is not sufficiently accurate, while Scilab's implementation
is much more accurate. 
\end{abstract}

\tableofcontents

\chapter{Introduction}

The Nelder-Mead simplex algorithm, published in 1965, is an enormously 
popular search method for multidimensional unconstrained optimization. 
The Nelder-Mead algorithm should not be confused with the (probably) 
more famous simplex algorithm of Dantzig for linear programming. The 
Nelder-Mead algorithm is especially popular in the fields of chemistry, 
chemical engineering, and medicine. Two measures of the ubiquity of the 
Nelder-Mead algorithm are that it appears in the best-selling handbook 
Numerical Recipes and in Matlab. In \cite{Torczon89multi-directionalsearch},
Virginia Torczon writes : "Margaret Wright has stated that over
fifty percent of the calls received by the support group for the NAG
software library concerned the version of the Nelder-Mead 
simplex algorithm to be found in that library". No derivative of the cost function is 
required, which makes the algorithm interesting for noisy problems.

The Nelder-Mead algorithm falls in the more general class of direct 
search algorithms. These methods use values of $f$ taken from a set of 
sample points and use that information to continue the sampling. The 
Nelder-Mead algorithm maintains a simplex which are approximations of an 
optimal point. The vertices are sorted according to the objective 
function values. The algorithm attemps to replace the worst vertex with 
a new point, which depends on the worst point and the centre of the best 
vertices. 

The goal of this toolbox is to provide a Nelder-Mead direct search optimization method to solve the 
following unconstrained optimization problem
\begin{eqnarray}
\min f(x)
\end{eqnarray}
where $x\in \RR^n$ where $n$ is the number of optimization parameters.
The Box complex algorithm, which is an extension of Nelder-Mead's algorithm, solves the 
following constrained problem
\begin{eqnarray}
&&\min f(x)\\
&&\ell_i \leq x_i \leq u_i, \qquad i = 1,n\\
&&g_j(x)\geq 0, \qquad j = 1, m\\
\end{eqnarray}
where $m$ is the number of nonlinear, positive constraints and $\ell_i,u_i\in \RR^n$ are the lower 
and upper bounds of the variables.

The Nelder-Mead algorithm may be used in the following optimization context :
\begin{itemize}
\item there is no need to provide the derivatives of the objective function,
\item the number of parameters is small (up to 10-20),
\item there are bounds and/or non linear constraints.
\end{itemize}

The internal design of the system is based on the following components :
\begin{itemize}
\item "neldermead" provides various Nelder-Mead variants and manages for Nelder-Mead specific settings, such as the method to compute the initial simplex, the specific termination criteria,
\item "fminsearch" provides a Scilab commands which aims at behaving as Matlab's fminsearch. Specific terminations criteria, initial simplex and auxiliary settings are automatically configured so that the behaviour of Matlab's fminsearch is exactly reproduced.
\item "optimset", "optimget" provides Scilab commands to emulate their Matlab counterparts.
\item "nmplot" provides a high-level component which provides directly output pictures for Nelder-Mead algorithm.
\end{itemize}
The current toolbox is based on (and therefore requires) the following toolboxes 
\begin{itemize}
\item "optimbase" provides an abstract class for a general optimization component, including the number of variables, the minimum and maximum bounds, the number of non linear inequality constraints, the loggin system, various termination criteria, the cost function, etc...
\item "optimsimplex" provides a class to manage a simplex made of an arbitrary number of vertices, including the computation of a simplex by various methods (axes, regular, Pfeffer's, randomized bounds), the computation of the size by various methods (diameter, sigma +, sigma-, etc...),
\end{itemize}

The following is a list of features the Nelder-Mead prototype algorithm currently provides :
\begin{itemize}
\item Manage various simplex initializations
  \begin{itemize}
  \item initial simplex given by user,
  \item initial simplex computed with a length and along the coordinate axes,
  \item initial regular simplex computed with Spendley et al. formula
  \item initial simplex computed by a small perturbation around the initial guess point
  \end{itemize}
\item Manage cost function
  \begin{itemize}
  \item optionnal additionnal argument
  \item direct communication of the task to perform : cost function or inequality constraints
  \end{itemize}
\item Manage various termination criteria, including maximum number of iterations, tolerance on function value (relative or absolute), 
  \begin{itemize}
  \item tolerance on x (relative or absolute),
  \item tolerance on standard deviation of function value (original termination criteria in [3]),
  \item maximum number of evaluations of cost function,
  \item absolute or relative simplex size,
  \end{itemize}
\item Manage the history of the convergence, including
  \begin{itemize}
  \item history of function values,
  \item history of optimum point,
  \item history of simplices,
  \item history of termination criterias,
  \end{itemize}
\item Provide a plot command which allows to graphically see the history of the simplices toward the optimum,
\item Provide query features for the status of the optimization process number of iterations, number of function evaluations, status of execution, function value at initial point, function value at optimal point, etc...
\item Spendley et al. fixed shaped algorithm,
\item Kelley restart based on simplex gradient,
\item O'Neill restart based on factorial search around optimum,
\item Box-like method managing bounds and nonlinear inequality constraints based on arbitrary number of vertices in the simplex.
\end{itemize}



\section{Quadratic equation}

In this section, we detail the computation of the roots of a quadratic polynomial.
As we shall see, there is a whole world from the mathematics formulas to the 
implementation of such computations. In the first part, we briefly report the formulas which allow to 
compute the real roots of a quadratic equation with real coefficients.
We then present the na�ve algorithm based on these mathematical formulas. 
In the second part, we make some experiments in Scilab and compare our
na�ve algorithm with the \emph{roots} Scilab primitive.
In the third part, we analyse 
why and how floating point numbers must be taken into account when the 
implementation of such roots is required.

\subsection{Theory}

We consider the following quadratic equation, with real 
coefficients $a, b, c \in \RR$ \cite{wikipediaquadratic,wikipedialossofsign,mathworldquadratic} :

\begin{eqnarray}
a x^2 + b x + c = 0.
\end{eqnarray}

The real roots of the quadratic equations are
\begin{eqnarray}
x_- &=& \frac{-b- \sqrt{b^2-4ac}}{2a}, \label{real:x-}\\
x_+ &=& \frac{-b+ \sqrt{b^2-4ac}}{2a}, \label{real:x+}
\end{eqnarray}
with the hypothesis that the discriminant $\Delta=b^2-4ac$
is positive.

The naive, simplified, algorithm which computes the roots of the 
quadratic is presented in figure \ref{naive-quadratic}.

\begin{figure}[htbp]
\begin{algorithmic}
\STATE $\Delta\gets b^2-4ac$
\STATE $s\gets \sqrt{\Delta}$
\STATE $x_-\gets (-b-s)/(2a)$
\STATE $x_+\gets (-b+s)/(2a)$
\end{algorithmic}
\caption{Naive algorithm to compute the real roots of a quadratic equation}
\label{naive-quadratic}
\end{figure}

\subsection{Experiments}

The following Scilab function is a straitforward implementation
of the previous formulas.

\lstset{language=Scilab}
\lstset{numbers=left}
\lstset{basicstyle=\footnotesize}
\lstset{keywordstyle=\bfseries}
\begin{lstlisting}
function r=myroots(p)
  c=coeff(p,0);
  b=coeff(p,1);
  a=coeff(p,2);
  r=zeros(2,1);
  r(1)=(-b+sqrt(b^2-4*a*c))/(2*a);
  r(2)=(-b-sqrt(b^2-4*a*c))/(2*a);
endfunction
\end{lstlisting}

The goal of this section is to show that some additionnal
work is necessary to compute the roots of the quadratic equation
with sufficient accuracy.
We will especially pay attention to rounding errors and 
overflow problems.
In this section, we show that the \emph{roots} command 
of the Scilab language is not \emph{naive}, in the sense that it 
takes into account for the floating point implementation details 
that we will see in the next section.



\subsubsection{Rounding errors}

We analyse the rounding errors which are 
appearing when the discriminant of the quadratic equation 
is such that $b^2\approx 4ac$.
We consider the following quadratic equation 
\begin{eqnarray}
\epsilon x^2 + (1/\epsilon)x - \epsilon = 0
\end{eqnarray}
with $\epsilon=0.0001=10^{-4}$.

The two real solutions of the quadratic equation are
\begin{eqnarray}
x_- &=& \frac{-1/\epsilon- \sqrt{1/\epsilon^2+4\epsilon^2}}{2\epsilon} \approx  -1/\epsilon^2, \\
x_+ &=& \frac{-1/\epsilon+ \sqrt{1/\epsilon^2+4\epsilon^2}}{2\epsilon} \approx  \epsilon^2
\end{eqnarray}

The following Scilab script shows an example of the computation
of the roots of such a polynomial with the \emph{roots}
primitive and with a naive implementation.
Only the positive root $x_+ \approx \epsilon^2$ is considered in this 
test (the $x_-$ root is so that $x_- \rightarrow -\infty$ in both 
implementations).

\lstset{language=Scilab}
\lstset{numbers=left}
\lstset{basicstyle=\footnotesize}
\lstset{keywordstyle=\bfseries}
\begin{lstlisting}
p=poly([-0.0001 10000.0 0.0001],"x","coeff");
e1 = 1e-8;
roots1 = myroots(p);
r1 = roots1(1);
roots2 = roots(p);
r2 = roots2(1);
error1 = abs(r1-e1)/e1;
error2 = abs(r2-e1)/e1;
printf("Expected : %e\n", e1);
printf("Naive method : %e (error=%e)\n", r1,error1);
printf("Scilab method : %e (error=%e)\n", r2, error2);
\end{lstlisting}

The script then prints out :

\begin{verbatim}
Expected : 1.000000e-008
Naive method : 9.094947e-009 (error=9.050530e-002)
Scilab method : 1.000000e-008 (error=1.654361e-016)
\end{verbatim}

The result is surprising, since the naive root has 
no correct digit and a relative error which is 14 orders 
of magnitude greater than the relative error of the Scilab root.

The explanation for such a behaviour is that the expression of the 
positive root is the following 

\begin{eqnarray}
x_+ &=& \frac{-1/\epsilon+ \sqrt{1/\epsilon^2+4\epsilon^2}}{2\epsilon}
\end{eqnarray}

and is numerically evalutated as 

\begin{verbatim}
\sqrt{1/\epsilon^2+4\epsilon^2} = 10000.000000000001818989
\end{verbatim}

As we see, the first digits are correct, but the last digits 
are polluted with rounding errors. When the expression $-1/\epsilon+ \sqrt{1/\epsilon^2+4\epsilon^2}$
is evaluated, the following computations are performed~:

\begin{verbatim}
-1/\epsilon+ \sqrt{1/\epsilon^2+4\epsilon^2} 
  = -10000.0 + 10000.000000000001818989 
  = 0.0000000000018189894035
\end{verbatim}

The user may think that the result is extreme, but it 
is not. Reducing furter the value of $\epsilon$ down to 
$\epsilon=10^{-11}$, we get the following output :

\begin{verbatim}
Expected : 1.000000e-022
Naive method : 0.000000e+000 (error=1.000000e+000)
Scilab method : 1.000000e-022 (error=1.175494e-016)
\end{verbatim}

The relative error is this time 16 orders of magnitude 
greater than the relative error of the Scilab root.
In fact, the naive implementation computes a false root $x_+$ even for 
a value of epsilon equal to $\epsilon=10^-3$, where the relative 
error is 7 times greater than the relative error produced by the 
\emph{roots} primitive.

\subsubsection{Overflow}

In this section, we analyse the overflow exception which is  
appearing when the discriminant of the quadratic equation 
is such that $b^2>> 4ac$.
We consider the following quadratic equation 
\begin{eqnarray}
x^2 + (1/\epsilon)x + 1 = 0
\end{eqnarray}
with $\epsilon\rightarrow 0$.

The roots of this equation are 
\begin{eqnarray}
x_- &\approx& -1/\epsilon \rightarrow -\infty, \qquad \epsilon \rightarrow 0\\
x_+ &\approx& -\epsilon \rightarrow 0^-, \qquad \epsilon \rightarrow 0
\end{eqnarray}
To create a difficult case, we search $\epsilon$ so that 
$1/\epsilon^2 = 10^{310}$, because we know that $10^{308}$
is the maximum value available with double precision floating 
point numbers. One possible solution is $\epsilon=10^{-155}$.

The following Scilab script shows an example of the computation
of the roots of such a polynomial with the \emph{roots}
primitive and with a naive implementation.

\lstset{language=Scilab}
\lstset{numbers=left}
\lstset{basicstyle=\footnotesize}
\lstset{keywordstyle=\bfseries}
\begin{lstlisting}
// Test #3 : overflow because of b
e=1.e-155
a = 1;
b = 1/e;
c = 1;
p=poly([c b a],"x","coeff");
expected = [-e;-1/e];
roots1 = myroots(p);
roots2 = roots(p);
error1 = abs(roots1-expected)/norm(expected);
error2 = abs(roots2-expected)/norm(expected);
printf("Expected : %e %e\n", expected(1),expected(2));
printf("Naive method : %e %e (error=%e)\n", roots1(1),roots1(2),error1);
printf("Scilab method : %e %e (error=%e)\n", roots2(1),roots2(2), error2);
\end{lstlisting}

The script then prints out :

\begin{verbatim}
Expected : -1.000000e-155 -1.000000e+155
Naive method : Inf Inf (error=Nan)
Scilab method : -1.000000e-155 -1.000000e+155 (error=0.000000e+000)
\end{verbatim}

As we see, the $b^2-4ac$ term has been evaluated as $1/\epsilon^2-4$,
which is approximately equal to $10^{310}$. This number cannot 
be represented in a floating point number. It therefore produces the 
IEEE overflow exception and set the result as \emph{Inf}.

\subsection{Explanations}

The technical report by G. Forsythe \cite{Forsythe1966} is 
especially interesting on that subject. The paper by Goldberg 
\cite{WhatEveryComputerScientist} is also a good reference for the 
quadratic equation. One can also consult the experiments performed by 
Nievergelt in \cite{Nievergelt2003}.

The following tricks are extracted from the 
\emph{quad} routine of the \emph{RPOLY} algorithm by
Jenkins \cite{Jenkins1975}. This algorithm is used by Scilab in the 
roots primitive, where a special case is handled when the 
degree of the equation is equal to 2, i.e. a quadratic equation.

\subsubsection{Properties of the roots}

One can easily show that the sum and the product of the roots
allow to recover the coefficients of the equation which was solve.
One can show that 
\begin{eqnarray}
x_- + x_+ &=&\frac{-b}{a}\\
x_- x_+ &=&\frac{c}{a}
\end{eqnarray}
Put in another form, one can state that the computed roots are 
solution of the normalized equation 
\begin{eqnarray}
x^2 - \left(\frac{x_- + x_+}{a}\right) x  + x_- x_+ &=&0
\end{eqnarray}

Other transformation leads to an alternative form for the roots. 
The original quadratic equation can be written as a quadratic 
equation on $1/x$
\begin{eqnarray}
c(1/x)^2 + b (1/x)  + a &=&0
\end{eqnarray}
Using the previous expressions for the solution of $ax^2+bx+c=0$ leads to the 
following expression of the roots of the quadratic equation when the 
discriminant is positive 
\begin{eqnarray}
x_- &=& \frac{2c}{-b+ \sqrt{b^2-4ac}}, \label{real:x-inverse}\\
x_+ &=& \frac{2c}{-b- \sqrt{b^2-4ac}} \label{real:x+inverse}
\end{eqnarray}
These roots can also be computed from \ref{real:x-}, with the 
multiplication by $-b+ \sqrt{b^2-4ac}$.

\subsubsection{Conditionning of the problem}

The conditionning of the problem may be evaluated with the 
computation of the partial derivatives of the roots of the 
equations with respect to the coefficients.
These partial derivatives measure the sensitivity of the 
roots of the equation with respect to small errors which might 
pollute the coefficients of the quadratic equations.

In the following, we note $x_-=\frac{-b- \sqrt{\Delta}}{2a}$ 
and $x_+=\frac{-b+ \sqrt{\Delta}}{2a}$ when $a\neq 0$.
If the discriminant is stricly positive and $a\neq 0$, i.e. if the roots 
of the quadratic are real, the partial derivatives of the 
roots are the following :
\begin{eqnarray}
\frac{\partial x_-}{\partial a} &=& \frac{c}{a\sqrt{\Delta}} + \frac{b+\sqrt{\Delta}}{2a^2}, \qquad a\neq 0, \qquad \Delta\neq 0\\
\frac{\partial x_+}{\partial a} &=& -\frac{c}{a\sqrt{\Delta}} + \frac{b-\sqrt{\Delta}}{2a^2}\\
\frac{\partial x_-}{\partial b} &=& \frac{-1-b/\sqrt{\Delta}}{2a}\\
\frac{\partial x_+}{\partial b} &=& \frac{-1+b/\sqrt{\Delta}}{2a}\\
\frac{\partial x_-}{\partial c} &=& \frac{1}{\sqrt{\Delta}}\\
\frac{\partial x_+}{\partial c} &=& -\frac{1}{\sqrt{\Delta}}
\end{eqnarray}

If the discriminant is zero, the partial derivatives of the 
double real root are the following :
\begin{eqnarray}
\frac{\partial x_\pm}{\partial a} &=& \frac{b}{2a^2}, \qquad a\neq 0\\
\frac{\partial x_\pm}{\partial b} &=& \frac{-1}{2a}\\
\frac{\partial x_\pm}{\partial c} &=& 0
\end{eqnarray}

The partial derivates indicate that if $a\approx 0$ or $\Delta\approx 0$,
the problem is ill-conditionned. 



\subsubsection{Floating-Point implementation : fixing rounding error}

In this section, we show how to compute the roots of a 
quadratic equation with protection against rounding 
errors, protection against overflow and a minimum 
amount of multiplications and divisions.

Few but important references deals with floating point
implementations of the roots of a quadratic polynomial.
These references include the important paper \cite{WhatEveryComputerScientist} by Golberg, 
the Numerical Recipes \cite{NumericalRecipes}, chapter 5, section 5.6
and \cite{FORSYTHE1991}, \cite{Nievergelt2003}, \cite{Kahan2004}.

The starting point is the mathematical solution of the quadratic equation, 
depending on the sign of the discriminant $\Delta=b^2 - 4ac$ :
\begin{itemize}
\item If $\Delta> 0$, there are two real roots, 
\begin{eqnarray}
x_\pm &=& \frac{-b\pm \sqrt{\Delta}}{2a}, \qquad a\neq 0
\end{eqnarray}
\item If $\Delta=0$, there are one double root,
\begin{eqnarray}
x_\pm &=& -\frac{b}{2a}, \qquad a\neq 0
\end{eqnarray}
\item If $\Delta< 0$, 
\begin{eqnarray}
x_\pm &=&\frac{-b}{2a} \pm i \frac{\sqrt{-\Delta}}{2a}, \qquad a\neq 0
\end{eqnarray}
\end{itemize}


In the following, we make the hypothesis that $a\neq 0$.

The previous experiments suggest that the floating point implementation
must deal with two different problems :
\begin{itemize}
\item rounding errors when $b^2\approx 4ac$ because of the cancelation of the 
terms which have opposite signs,
\item overflow in the computation of the discriminant $\Delta$ when $b$ is 
large in magnitude with respect to $a$ and $c$.
\end{itemize}

When $\Delta>0$, the rounding error problem can be splitted in two cases
\begin{itemize}
\item if $b<0$, then $-b+\sqrt{b^2-4ac}$ may suffer of rounding errors,
\item if $b>0$, then $-b-\sqrt{b^2-4ac}$ may suffer of rounding errors.
\end{itemize}
 
Obviously, the rounding problem will not appear when $\Delta<0$,
since the complex roots do not use the sum $-b+\sqrt{b^2-4ac}$.
When $\Delta=0$, the double root does not cause further trouble.
The rounding error problem must be solved only when $\Delta>0$ and the 
equation has two real roots.

A possible solution may found in combining the following expressions for the 
roots 
\begin{eqnarray}
x_- &=& \frac{-b- \sqrt{b^2-4ac}}{2a}, \label{real:x-2}\\
x_- &=& \frac{2c}{-b+ \sqrt{b^2-4ac}}, \label{real:x-inverse2}\\
x_+ &=& \frac{-b+ \sqrt{b^2-4ac}}{2a}, \label{real:x+2}\\
x_+ &=& \frac{2c}{-b- \sqrt{b^2-4ac}} \label{real:x+inverse2}
\end{eqnarray}

The trick is to pick the formula so that the sign of $b$ is the 
same as the sign of the square root.

The following choice allow to solve the rounding error problem 
\begin{itemize}
\item compute $x_-$ : if $b<0$, then compute $x_-$ from \ref{real:x-inverse2}, else 
(if $b>0$), compute $x_-$ from \ref{real:x-2},
\item compute $x_+$ : if $b<0$, then compute $x_+$ from \ref{real:x+2}, else 
(if $b>0$), compute $x_+$ from \ref{real:x+inverse2}.
\end{itemize}

The solution of the rounding error problem  can be adressed, by considering the 
modified Fagnano formulas
\begin{eqnarray}
x_1 &=& -\frac{2c}{b+sgn(b)\sqrt{b^2-4ac}}, \\
x_2 &=& -\frac{b+sgn(b)\sqrt{b^2-4ac}}{2a}, 
\end{eqnarray}
where 
\begin{eqnarray}
sgn(b)=\left\{\begin{array}{l}
1, \textrm{ if } b\geq 0,\\
-1, \textrm{ if } b< 0,
\end{array}\right.
\end{eqnarray}
The roots $x_{1,2}$ correspond to $x_{+,-}$ so that if $b<0$, $x_1=x_-$ and
if $b>0$, $x_1=x_+$. On the other hand, if $b<0$, $x_2=x_+$ and
if $b>0$, $x_2=x_-$.

An additionnal remark is that the division by two (and the multiplication
by 2) is exact with floating point numbers so these operations
cannot be a source of problem. But it is 
interesting to use $b/2$, which involves only one division, instead
of the three multiplications $2*c$, $2*a$ and $4*a*c$.
This leads to the following expressions of the real roots 
\begin{eqnarray}
x_- &=& -\frac{c}{(b/2)+sgn(b)\sqrt{(b/2)^2-ac}}, \\
x_+ &=& -\frac{(b/2)+sgn(b)\sqrt{(b/2)^2-ac}}{a}, 
\end{eqnarray}
which can be simplified into
\begin{eqnarray}
b'&=&b/2\\
h&=& -\left(b'+sgn(b)\sqrt{b'^2-ac}\right)\\
x_1 &=& \frac{c}{h}, \\
x_2 &=& \frac{h}{a}, 
\end{eqnarray}
where the discriminant is positive, i.e. $b'^2-ac>0$.

One can use the same value $b'=b/2$ with the complex roots in the 
case where the discriminant is negative, i.e. $b'^2-ac<0$ :
\begin{eqnarray}
x_1 &=& -\frac{b'}{a} - i \frac{\sqrt{ac-b'^2}}{a}, \\
x_2 &=& -\frac{b'}{a} + i \frac{\sqrt{ac-b'^2}}{a}, 
\end{eqnarray}

A more robust algorithm, based on the previous analysis is presented in figure \ref{robust-quadratic}.
By comparing \ref{naive-quadratic} and \ref{robust-quadratic}, we can see that 
the algorithms are different in many points.

\begin{figure}[htbp]
\begin{algorithmic}
\IF {$a=0$}
        \IF {$b=0$}
            \STATE $x_-\gets 0$
            \STATE $x_+\gets 0$
        \ELSE
            \STATE $x_-\gets -c/b$
            \STATE $x_+\gets 0$
        \ENDIF
\ELSIF {$c=0$}
        \STATE $x_-\gets -b/a$
        \STATE $x_+\gets 0$        
\ELSE
        \STATE $b'\gets b/2$
        \STATE $\Delta\gets b'^2 - ac$
        \IF {$\Delta<0$}
                \STATE $s\gets \sqrt{-\Delta}$
                \STATE $x_1^R\gets -b'/a$
                \STATE $x_1^I\gets -s/a$
                \STATE $x_2^R\gets x_-^R$
                \STATE $x_2^I\gets -x_1^I$
        \ELSIF {$\Delta=0$}
                \STATE $x_1\gets -b'/a$
                \STATE $x_2\gets x_2$
        \ELSE
                \STATE $s\gets \sqrt{\Delta}$
                \IF {$b>0$}
                    \STATE $g=1$
                \ELSE
                    \STATE $g=-1$
                \ENDIF
                \STATE $h=-(b'+g*s)$
                \STATE $x_1\gets c/h$
                \STATE $x_2\gets h/a$
        \ENDIF
\ENDIF 
\end{algorithmic}
\caption{A more robust algorithm to compute the roots of a quadratic equation}
\label{robust-quadratic}
\end{figure}

\subsubsection{Floating-Point implementation : fixing overflow problems}

The remaining problem is to compute $b'^2-ac$ without creating 
unnecessary overflows. 

Notice that a small improvment
has allread been done : if $|b|$ is close to the upper bound $10^{154}$, 
then $|b'|$ may be less difficult to process since $|b'|=|b|/2 < |b|$.
One can then compute the square root by using normalization methods, 
so that the overflow problem can be drastically reduced.
The method is based on the fact that the term $b'^2-ac$ can be 
evaluted with two equivalent formulas
\begin{eqnarray}
b'^2-ac &=& b'^2\left[1-(a/b')(c/b')\right] \\
b'^2-ac &=& c\left[b'(b'/c) - a\right]
\end{eqnarray}

\begin{itemize}
\item If $|b'|>|c|>0$, then the expression involving $\left(1-(a/b')(c/b')\right)$
is so that no overflow is possible since $|c/b'| < 1$ and the problem occurs
only when $b$ is large in magnitude with respect to $a$ and $c$.
\item If $|c|>|b'|>0$, then the expression involving $\left(b'(b'/c) - a\right)$
should limit the possible overflows since $|b'/c| < 1$.
\end{itemize}
These normalization tricks are similar to the one used by Smith in the 
algorithm for the division of complex numbers \cite{Smith1962}.







\section{Numerical derivatives}

In this section, we detail the computation of the numerical derivative of 
a given function.

In the first part, we briefly report the first order forward formula, which 
is based on the Taylor theorem.
We then present the na�ve algorithm based on these mathematical formulas. 
In the second part, we make some experiments in Scilab and compare our
na�ve algorithm with the \emph{derivative} Scilab primitive.
In the third part, we analyse 
why and how floating point numbers must be taken into account when the 
numerical derivatives are to compute.

A reference for numerical derivates 
is \cite{AbramowitzStegun1972}, chapter 25. "Numerical Interpolation, 
Differentiation and Integration" (p. 875).
The webpage \cite{schimdtnd} and the book \cite{NumericalRecipes} give
results about the rounding errors.

\subsection{Theory}

The basic result is the Taylor formula with one variable \cite{dixmier}

\begin{eqnarray}
f(x+h) &=& f(x) 
+ h f^\prime(x)
+\frac{h^2}{2} f^{\prime \prime}(x)
+\frac{h^3}{6} f^{\prime \prime \prime}(x)
+\frac{h^4}{24} f^{\prime \prime \prime \prime}(x) + \mathcal{O}(h^5)
\end{eqnarray}

If we write the Taylor formulae of a one variable function $f(x)$ 
\begin{eqnarray}
f(x+h) &\approx& f(x) + h \frac{\partial f}{\partial x}+ \frac{h^2}{2} f^{\prime \prime}(x)
\end{eqnarray}
we get the forward difference which approximates the first derivate at order 1 
\begin{eqnarray}
f^\prime(x) &\approx& \frac{f(x+h)  - f(x)}{h} + \frac{h}{2} f^{\prime \prime}(x)
\end{eqnarray}

The naive algorithm to compute the numerical derivate of 
a function of one variable is presented in figure \ref{naive-numericalderivative}.

\begin{figure}[htbp]
\begin{algorithmic}
\STATE $f'(x) \gets (f(x+h)-f(x))/h$
\end{algorithmic}
\caption{Naive algorithm to compute the numerical derivative of a function of one variable}
\label{naive-numericalderivative}
\end{figure}

\subsection{Experiments}

The following Scilab function is a straitforward implementation
of the previous algorithm.

\lstset{language=Scilab}
\lstset{numbers=left}
\lstset{basicstyle=\footnotesize}
\lstset{keywordstyle=\bfseries}
\begin{lstlisting}
function fp = myfprime(f,x,h)
  fp = (f(x+h) - f(x))/h;
endfunction
\end{lstlisting}

In our experiments, we will compute the derivatives of the 
square function $f(x)=x^2$, which is $f'(x)=2x$.
The following Scilab script implements the square function.

\lstset{language=Scilab}
\lstset{numbers=left}
\lstset{basicstyle=\footnotesize}
\lstset{keywordstyle=\bfseries}
\begin{lstlisting}
function y = myfunction (x)
  y = x*x;
endfunction
\end{lstlisting}

The most na�ve idea is that the computed relative error 
is small when the step $h$ is small. Because \emph{small}
is not a priori clear, we take $\epsilon\approx 10^{-16}$
in double precision as a good candidate for \emph{small}.
In the following script, we compare the computed 
relative error produced by our na�ve method with step
$h=\epsilon$ and the \emph{derivative} primitive with
default step.

\lstset{language=Scilab}
\lstset{numbers=left}
\lstset{basicstyle=\footnotesize}
\lstset{keywordstyle=\bfseries}
\begin{lstlisting}
x = 1.0;
fpref = derivative(myfunction,x,order=1);
e = abs(fpref-2.0)/2.0;
mprintf("Scilab f''=%e, error=%e\n", fpref,e);
h = 1.e-16;
fp = myfprime(myfunction,x,h);
e = abs(fp-2.0)/2.0;
mprintf("Naive f''=%e, h=%e, error=%e\n", fp,h,e);
\end{lstlisting}

When executed, the previous script prints out :

\begin{verbatim}
Scilab f'=2.000000e+000, error=7.450581e-009
Naive f'=0.000000e+000, h=1.000000e-016, error=1.000000e+000
\end{verbatim}

Our na�ve method seems to be quite inaccurate and has not 
even 1 significant digit !
The Scilab primitive, instead, has 9 significant digits.

Since our faith is based on the truth of the mathematical
theory, some deeper experiments must be performed.
We then make the following experiment, by taking an
initial step $h=1.0$ and then dividing $h$ by 10 at each
step of a loop with 20 iterations.

\lstset{language=Scilab}
\lstset{numbers=left}
\lstset{basicstyle=\footnotesize}
\lstset{keywordstyle=\bfseries}
\begin{lstlisting}
x = 1.0;
fpref = derivative(myfunction,x,order=1);
e = abs(fpref-2.0)/2.0;
mprintf("Scilab f''=%e, error=%e\n", fpref,e);
h = 1.0;
for i=1:20
  h=h/10.0;
  fp = myfprime(myfunction,x,h);
  e = abs(fp-2.0)/2.0;
  mprintf("Naive f''=%e, h=%e, error=%e\n", fp,h,e);
end
\end{lstlisting}

Scilab then produces the following output.

\begin{verbatim}
Scilab f'=2.000000e+000, error=7.450581e-009
Naive f'=2.100000e+000, h=1.000000e-001, error=5.000000e-002
Naive f'=2.010000e+000, h=1.000000e-002, error=5.000000e-003
Naive f'=2.001000e+000, h=1.000000e-003, error=5.000000e-004
Naive f'=2.000100e+000, h=1.000000e-004, error=5.000000e-005
Naive f'=2.000010e+000, h=1.000000e-005, error=5.000007e-006
Naive f'=2.000001e+000, h=1.000000e-006, error=4.999622e-007
Naive f'=2.000000e+000, h=1.000000e-007, error=5.054390e-008
Naive f'=2.000000e+000, h=1.000000e-008, error=6.077471e-009
Naive f'=2.000000e+000, h=1.000000e-009, error=8.274037e-008
Naive f'=2.000000e+000, h=1.000000e-010, error=8.274037e-008
Naive f'=2.000000e+000, h=1.000000e-011, error=8.274037e-008
Naive f'=2.000178e+000, h=1.000000e-012, error=8.890058e-005
Naive f'=1.998401e+000, h=1.000000e-013, error=7.992778e-004
Naive f'=1.998401e+000, h=1.000000e-014, error=7.992778e-004
Naive f'=2.220446e+000, h=1.000000e-015, error=1.102230e-001
Naive f'=0.000000e+000, h=1.000000e-016, error=1.000000e+000
Naive f'=0.000000e+000, h=1.000000e-017, error=1.000000e+000
Naive f'=0.000000e+000, h=1.000000e-018, error=1.000000e+000
Naive f'=0.000000e+000, h=1.000000e-019, error=1.000000e+000
Naive f'=0.000000e+000, h=1.000000e-020, error=1.000000e+000
\end{verbatim}

We see that the relative error begins by decreasing, and then 
is increasing.
Obviously, the optimum step is approximately $h=10^{-8}$, where the
relative error is approximately $e_r=6.10^{-9}$. 
We should not be surprised to see that Scilab has computed 
a derivative which is near the optimum.

\subsection{Explanations}

\subsubsection{Floating point implementation}

With a floating point computer, the total 
error that we get from the forward difference approximation
is (skipping the multiplication constants) the sum of the 
linearization error $E_l = h$ (i.e. the $\mathcal{O}(h)$ term)
and the rounding error $rf(x)$ on the difference $f(x+h)  - f(x)$
\begin{eqnarray}
E = \frac{rf(x)}{h} + \frac{h}{2} f^{\prime \prime}(x)
\end{eqnarray}
When $h\rightarrow \infty$, the error is then the sum of a
term which converges toward $+\infty$ and a term which converges toward 0.
The total error is minimized when both terms are equal.
With a single precision computation, the rounding error is $r = 10^{-7}$
and with a double precision computation, the rounding error is $r = 10^{-16}$.
We make here the assumption that the values $f(x)$ and 
$f^{\prime \prime}(x)$ are near 1 so that the error can be written 
\begin{eqnarray}
E = \frac{r}{h} + h
\end{eqnarray}
We want to compute the step $h$ from the rounding error $r$ with a 
step satisfying 
\begin{eqnarray}
h = r^\alpha
\end{eqnarray}
for some $\alpha > 0$.
The total error is therefore 
\begin{eqnarray}
E = r^{1-\alpha} + r^\alpha
\end{eqnarray}
The total error is minimized when both terms are equal, that is, 
when the exponents are equal  $1-\alpha = \alpha$ which leads to 
\begin{eqnarray}
\alpha = \frac{1}{2}
\end{eqnarray}
We conclude that the step which minimizes the error is 
\begin{eqnarray}
h = r^{1/2}
\end{eqnarray}
and the associated error is 
\begin{eqnarray}
E = 2 r^{1/2}
\end{eqnarray}

Typical values with single precision are $h = 10^{-4}$ and $E=2. 10^{-4}$
and with double precision $h = 10^{-8}$ and $E=2. 10^{-8}$.
These are the minimum error which are achievable with a forward difference
numerical derivate.

To get a significant value of the step $h$, the step is computed
with respect to the point where the derivate is to compute
\begin{eqnarray}
h = r^{1/2} x
\end{eqnarray}

One can generalize the previous computation with the 
assumption that the scaling parameter from the Taylor 
expansion is $h^{\alpha_1}$ and the order of the formula
is $\mathcal{O}(h^{\alpha_2})$. The total error is then 
\begin{eqnarray}
E = \frac{r}{h^{\alpha_1}} + h^{\alpha_2}
\end{eqnarray}
The optimal step is then 
\begin{eqnarray}
h = r^{\frac{1}{\alpha_1 + \alpha_2}}
\end{eqnarray}
and the associated error is 
\begin{eqnarray}
E = 2 r^{\frac{\alpha_2}{\alpha_1 + \alpha_2}}
\end{eqnarray}


An additional trick \cite{NumericalRecipes} is to compute the 
step $h$ so that the rounding error for the sum $x+h$ is minimum.
This is performed by the following algorithm, which implies a temporary 
variable $t$
\begin{eqnarray}
t = x + h\\
h = t - h
\end{eqnarray}


\subsubsection{Results}

In the following results, the variable $x$ is either a 
scalar $x^in \RR$ or a vector $x\in \RR^n$.
When $x$ is a vector, the step $h_i$ is defined by
\begin{eqnarray}
h_i = (0,\ldots,0,1,0,\ldots,0)
\end{eqnarray}
so that the only non-zero component of $h_i$ is the $i$-th component.

\begin{itemize}

\item First derivate : forward 2 points 
\begin{eqnarray}
f^\prime(x) &\approx& \frac{f(x+h)  - f(x)}{h} + \mathcal{O}(h)
\end{eqnarray}
Optimal step : $h = r^{1/2}$ and error $E=2r^{1/2}$.\\
Single precision : $h \approx 10^{-4}$ and $E\approx 10^{-4}$.\\
Double precision $h \approx 10^{-8}$ and $E\approx 10^{-8}$.

\item First derivate : backward 2 points 
\begin{eqnarray}
f^\prime(x) &\approx& \frac{f(x) - f(x-h)}{h} + \mathcal{O}(h)
\end{eqnarray}
Optimal step : $h = r^{1/2}$ and error $E=2r^{1/2}$.\\
Single precision : $h \approx 10^{-4}$ and $E\approx 10^{-4}$.\\
Double precision $h \approx 10^{-8}$ and $E\approx 10^{-8}$.

\item First derivate : centered 2 points 
\begin{eqnarray}
f^\prime(x) &\approx& \frac{f(x+h) - f(x-h)}{2h} + \mathcal{O}(h^2)
\end{eqnarray}
Optimal step : $h = r^{1/3}$ and error $E=2r^{2/3}$.\\
Single precision : $h \approx 10^{-3}$ and $E\approx 10^{-5}$.\\
Double precision $h \approx 10^{-5}$ and $E\approx 10^{-10}$.

\end{itemize}

\subsubsection{Robust algorithm}

The robust algorithm to compute the numerical derivate of 
a function of one variable is presented in figure \ref{robust-numericalderivative}.

\begin{figure}[htbp]
\begin{algorithmic}
\STATE $h \gets \sqrt{\epsilon}$
\STATE $f'(x) \gets (f(x+h)-f(x))/h$
\end{algorithmic}
\caption{A more robust algorithm to compute the numerical derivative of a function of one variable}
\label{robust-numericalderivative}
\end{figure}

\subsection{One step further}

In this section, we analyse the behaviour of \emph{derivative}
when the point $x$ is either large $x \rightarrow \infty$, 
when $x$ is small $x \rightarrow 0$ and when $x = 0$.
We compare these results with the \emph{numdiff} command,
which does not use the same step strategy. As we are going 
to see, both commands performs the same when $x$ is near 1, but 
performs very differently when x is large or small.

We have allready explained the theory of the floating 
point implementation of the \emph{derivative} command.
Is it completely \emph{bulletproof} ? Not exactly. 

See for example the following Scilab session, where one computes the 
numerical derivative of $f(x)=x^2$ for $x=10^{-100}$. The 
expected result is $f'(x) = 2. \times 10^{-100}$.

\begin{verbatim}
-->fp = derivative(myfunction,1.e-100,order=1)
 fp  =
    0.0000000149011611938477  
-->fe=2.e-100
 fe  =
    2.000000000000000040-100  
-->e = abs(fp-fe)/fe
 e  =
    7.450580596923828243D+91  
\end{verbatim}

The result does not have any significant digits.

The explanation is that the step is computed with $h = \sqrt{eps}\approx 10^{-8}$.
Then $f(x+h)=f(10^{-100} + 10^{-8}) \approx f(10^{-8}) = 10^{-16}$, because the 
term $10^{-100}$ is much smaller than $10^{-8}$. The result of the 
computation is therefore $(f(x+h) - f(x))/h = (10^{-16} + 10^{-200})/10^{-8} \approx 10^{-8}$.

The additionnal experiment 

\begin{verbatim}
-->sqrt(%eps)
 ans  =
    0.0000000149011611938477  
\end{verbatim}

allows to check that the result of the computation simply is $\sqrt{eps}$.
That experiment shows that the \emph{derivative} command uses a 
wrong defaut step $h$ when $x$ is very small.

To improve the accuracy of the computation, one can take control of the 
step $h$. A reasonable solution is to use $h=\sqrt{\epsilon}|x|$ so that the 
step is scaled depending on $x$. 
The following script illustrates than method, which produces 
results with 8 significant digits.

\begin{verbatim}
-->fp = derivative(myfunction,1.e-100,order=1,h=sqrt(%eps)*1.e-100)
 fp  =
    2.000000013099139394-100  
-->fe=2.e-100
 fe  =
    2.000000000000000040-100  
-->e = abs(fp-fe)/fe
 e  =
    0.0000000065495696770794  
\end{verbatim}

But when $x$ is exactly zero, the scaling method cannot work, because 
it would produce the step $h=0$, and therefore a division by zero
exception. In that case, the default step provides a good accuracy.

Another command is available in Scilab to compute the 
numerical derivatives of a given function, that is \emph{numdiff}.
The \emph{numdiff} command uses the step 
\begin{eqnarray}
h=\sqrt{\epsilon}(1+10^{-3}|x|).
\end{eqnarray}
In the following paragraphs, we try to analyse why this formula 
has been chosen. As we are going to check experimentally, this step
formula performs better than \emph{derivative} when $x$ is 
large.

As we can see the following session, the behaviour is approximately 
the same when the value of $x$ is 1.

\begin{verbatim}
-->fp = numdiff(myfunction,1.0)
 fp  =
    2.0000000189353417390237  
-->fe=2.0
 fe  =
    2.  
-->e = abs(fp-fe)/fe
 e  =
    9.468D-09  
\end{verbatim}

The accuracy is slightly decreased with respect to the optimal
value 7.450581e-009 which was produced by derivative. But the number 
of significant digits is approximately the same, i.e. 9 digits.

The goal of this step is to produce good accuracy when the value of $x$ 
is large, where the \emph{numdiff} command produces accurate
results, while \emph{derivative} performs poorly.

\begin{verbatim}
-->numdiff(myfunction,1.e10)
  ans  =
    2.000D+10  
-->derivative(myfunction,1.e10,order=1)
 ans  =
    0.  
\end{verbatim}

This step is a trade-off because it allows to keep a good accuracy with
large values of $x$, but produces a slightly sub-optimal step size when 
$x$ is near 1. The behaviour near zero is the same, i.e. both commands 
produce wrong results when $x \rightarrow 0$ and $x\neq 0$.



\section{Complex division}

Dans cette partie, nous analysons le probl�me de la division complexe 
dans Scilab. Nous mettons en lumi�re pourquoi le traitement num�rique interne 
de Scilab est parfois inutilis� et pourquoi il est parfois inutile, voire 
nuisible (rappelons que le but d'une introduction est d'attirer le lecteur
vers la suite du texte, raison pour laquelle nous avons r�dig� ces mots 
les plus provocants possible !).

Nous d�taillons en particulier la diff�rence entre 
d�finition math�matique et impl�mentation en nombres flottants.
Nous montrons comment la division de nombres complexes est 
effectu� dans Scilab lorsque l'op�rateur "/" est utilis�.
Nous montrons �galement que l'impl�mentation n'est pas 
utilis� de mani�re consistante dans Scilab.
Nous montrerons enfin que les librairies utilis�es dans 
les compilateurs Intel et gfortran traitent le probl�me.

\subsection{Theory}

\subsubsection{Algebraic computations}

Il est de notori�t� publique que la formule math�matique 
qui permet de calculer la division entre deux nombres complexes 
\begin{eqnarray}
\frac{a + ib}{c + id} = \frac{ac + bd}{c^2 + d^2} + i \frac{bc - ad}{c^2 + d^2} 
\end{eqnarray}

\subsubsection{Naive algorithm}

The naive algorithm for the computation of the complex division
is presented in figure \ref{naive-complexdivision}.

\begin{figure}[htbp]
\begin{algorithmic}
\STATE $den \gets c^2 + d^2$
\STATE $e \gets (ac + bd)/ den$
\STATE $f \gets (bc - ad)/ den$ 
\end{algorithmic}
\caption{Naive algorithm to compute the complex division}
\label{naive-complexdivision}
\end{figure}


\subsection{Experiments}
n'est pas robuste quand on utilise des nombres flottants \cite{368661}.
En r�sum�, le probl�me arrive lorsque les nombres a, b, c ou d s'approchent
du domaine de d�finition des nombres flottants, ce qui peut provoquer
un overflow ou un underflow.

Supposons que l'on traite la division suivante 

\begin{eqnarray}
\frac{1 + I * 1}{1 + I * 1e308} = 1e308 - I * 1e-308
\end{eqnarray}

Utilisons la formule math�matique, na�ve, pour v�rifier cette division. 

\begin{eqnarray}
den &=& c� + d� = 1� + (1e308)� = 1 + 1e616 ~1e616 \\
e &=& (ac + bd)/ den = (1*1 + 1*1e308)/1e616 ~ 1e308/1e616 ~ 1e-308\\
f &=& (bc - ad)/ den = (1*1 - 1*1e308)/1e616 ~ -1e308/1e616 ~ -1e-308
\end{eqnarray}

Num�riquement, les choses ne se passent pas ainsi, car vmax = 1e308 est la 
valeur maximale avant overflow. Plus pr�cis�ment, vmax est tel que 
tout nombre v > vmax satisfait 
\begin{eqnarray}
(v * 2) / 2 != v
\end{eqnarray}
Autrement dit, la multiplication et la division ne sont plus pr�cises pour 
des nombres v > vmax.

Si on utilise la formule na�ve avec des nombres double precision, alors
\begin{eqnarray}
den = c� + d� = 1� + (1e308)� = Inf
\end{eqnarray}
c'est � dire un overflow. Les termes e et f sont alors calcul�s par
\begin{eqnarray}
e = (ac + bd)/ den = (1*1 + 1*1e308)/Inf = 1e308/Inf = 0\\
f = (bc - ad)/ den = (1*1 - 1*1e308)/Inf = -1e308/Inf = 0
\end{eqnarray}
Le r�sultat est alors faux sur le plan math�matique, mais surtout, il 
est impr�cis sur le plan num�rique, dans la mesure o� aucun des 
nombres initiaux n'�tait sup�rieur � vmax.

On peut �galement montrer que la formule na�ve peut g�n�rer 
des underflow.

Supposons que l'on veuille calculer la division suivante :
\begin{eqnarray}
\frac{1 + I * 1}{1e-308 + I * 1e-308}= 1e308
\end{eqnarray}

Utilisons la formule math�matique, na�ve, pour v�rifier cette division. 

\begin{eqnarray}
den &=& c� + d� = (1e-308)� + (1e-308)� = 1e-616 + 1e-616 = 2e-616 \\
e &=& (ac + bd)/ den = (1*1e-308 + 1*1e-308)/2e-616 ~ 2e-308/2e-616 ~ 1e308 \\
f &=& (bc - ad)/ den = (1*1e-308 - 1*1e-308)/2e-616 ~ 0/1e-616 ~ 0
\end{eqnarray}

Avec des nombres double precision, le calcul ne se passe pas ainsi.
Le d�nominateur va provoquer un underflow et va �tre num�riquement
mis � z�ro, de telle sorte que 

\begin{eqnarray}
den &=& c� + d� = (1e-308)� + (1e-308)� = 1e-616 + 1e-616 = 0 \\
e &=& (ac + bd)/ den = (1*1e-308 + 1*1e-308)/0 ~ 2e-308/0 ~ Inf \\
f &=& (bc - ad)/ den = (1*1e-308 - 1*1e-308)/0 ~ 0/0 ~ NaN \\
\end{eqnarray}

\subsubsection{Scilab implementation}

On peut exp�rimenter facilement dans la console Scilab v5.0.2, prenant soin
de choisir un cas test qui pose probl�me.

\begin{verbatim}
a=1+%i*1
b=1+%i*1e308
a/b
\end{verbatim}

Le test montre que l'impl�mentation dans Scilab correspond � celle 
de Smith, puisque, dans ce cas, le r�sultat est correct :

\begin{verbatim}
-->a=1+%i*1
 a  =
 
    1. + i    
 
-->b=1+%i*1e308
 b  =
 
    1. + 1.000+308i  
 
-->a/b
 ans  =
 
    1.000-308 - 1.000-308i  
\end{verbatim}

Si on effectue le calcul pas � pas, on trouve que le 
code utilis� correspond au code source de "wwdiv" du module
elementary\_functions :

\begin{verbatim}
subroutine wwdiv(ar, ai, br, bi, cr, ci, ierr)
\end{verbatim}

qui impl�mente la formule de Smith.
Or, comme on l'a vu, l'algorithme de Smith poss�de une robustesse limit�e.

On peut �galement tester les limites de la m�thode de Smith, en 
exp�rimentant de la mani�re suivante :

\begin{verbatim}
-->a = 1e307 + %i * 1e-307
 a  =
    1.000+307 + 1.000-307i  
-->b = 1e205 + %i * 1e-205
 b  =
    1.000+205 + 1.000-205i  
-->a/b
 ans  =
    1.000+102  
\end{verbatim}

Cette r�ponse est fausse, mais correspond bien au r�sultat 
de la m�thode de Smith.

Par ailleurs, il est int�ressant de constater que la proc�dure wwdiv n'est 
pas utilis�e syst�matiquement dans Scilab.
En effet, on peut trouver des sections de code utilisant l'impl�mentation
de la division complexe fournie par le compilateur.

Par exemple, la fonction lambda = spec(A,B) calcule les valeurs propres
g�n�ralis�es des matrices complexes A et B. L'interface vers la fonction
zggev de Lapack est impl�ment�e dans intzggev, dans laquelle on peut trouver 
les lignes suivantes 

\begin{verbatim}
          do 15 i = 1, N
            zstk(lALPHA-1+i)=zstk(lALPHA-1+i)/zstk(lBETA-1+i)
 15      continue
\end{verbatim}

Ce morceau de code permet de stocker dans le tableau complex 
associ� � la variable alpha le r�sultat de la division de alpha/beta.
Comme les deux op�randes sont de type complexe, c'est le compilateur
fortran qui impl�mente la division (et, bien s�r, sans utiliser wwdiv).


\subsubsection{Fortran code}

Le code fortran suivant permet d'illustrer l'ensemble des points 
pr�sent�s pr�c�dement. On le test avec le compilateur Intel Fortran 10.1.

\begin{verbatim}

      program rndof
      complex*16 a 
      complex*16 b
      complex*16 c
      double precision ar
      double precision ai
      double precision br
      double precision bi
      double precision cr
      double precision ci
c Check that naive implementation does not have a bug
      ar = 1.d0
      ai = 2.d0
      br = 3.d0
      bi = 4.d0
      call compare(ar, ai, br, bi)
c Check that naive implementation is not robust with respect to overflow
      ar = 1.d0
      ai = 1.d0
      br = 1.d0
      bi = 1.d308
      call compare(ar, ai, br, bi)
c Check that naive implementation is not robust with respect to underflow
      ar = 1.d0
      ai = 1.d0
      br = 1.d-308
      bi = 1.d-308
      call compare(ar, ai, br, bi)
c Check that Smith implementation is not robust with respect to complicated underflow
      ar = 1.d307
      ai = 1.d-307
      br = 1.d205
      bi = 1.d-205
      call compare(ar, ai, br, bi)
      end

      subroutine naive(ar, ai, br, bi, cr, ci)
      double precision, intent(in) :: ar
      double precision, intent(in) :: ai
      double precision, intent(in) :: br
      double precision, intent(in) :: bi
      double precision, intent(out) :: cr
      double precision, intent(out) :: ci
      double precision den
      den = br * br + bi * bi
      cr = (ar * br + ai * bi) / den
      ci = (ai * br - ar * bi) / den
      end
      
      subroutine smith(ar, ai, br, bi, cr, ci)
      double precision, intent(in) :: ar
      double precision, intent(in) :: ai
      double precision, intent(in) :: br
      double precision, intent(in) :: bi
      double precision, intent(out) :: cr
      double precision, intent(out) :: ci
      double precision den
         if (abs(br) .ge. abs(bi)) then
            r = bi / br
            den = br + r*bi
            cr = (ar + ai*r) / den
            ci = (ai - ar*r) / den
         else
            r = br / bi
            den = bi + r*br
            cr = (ar*r + ai) / den
            ci = (ai*r - ar) / den
         endif
      end
      
      subroutine compare(ar, ai, br, bi)
      double precision, intent(in) :: ar
      double precision, intent(in) :: ai
      double precision, intent(in) :: br
      double precision, intent(in) :: bi
      complex*16 a 
      complex*16 b
      complex*16 c
      double precision cr
      double precision ci
      print *, "****************"
      call naive(ar, ai, br, bi, cr, ci)
      print *, "Naive   :", cr, ci
      call smith(ar, ai, br, bi, cr, ci)
      print *, "Smith   :", cr, ci
      a= dcmplx(ar, ai) 
      b = dcmplx(br, bi) 
      c = a/b
      print * , "Fortran:", c
      end
      
\end{verbatim}

Si on compile ce code avec Intel Fortran 10.1, on obtient 
l'affichage suivant dans la console.

\begin{verbatim}
 ****************
 c naive:  0.440000000000000       8.000000000000000E-002
 c Smith:  0.440000000000000       8.000000000000000E-002
 c Fortran: (0.440000000000000,8.000000000000000E-002)
 ****************
 c naive:  0.000000000000000E+000  0.000000000000000E+000
 c Smith:  9.999999999999999E-309 -9.999999999999999E-309
 c Fortran: (9.999999999999999E-309,-9.999999999999999E-309)
 ****************
 c naive: Infinity                NaN
 c Smith:  1.000000000000000E+308  0.000000000000000E+000
 c Fortran: (1.000000000000000E+308,0.000000000000000E+000)
 ****************
 c naive:  0.000000000000000E+000  0.000000000000000E+000
 c Smith:  1.000000000000000E+102  0.000000000000000E+000
 c Fortran: (9.999999999999999E+101,-9.999999999999999E-309)
\end{verbatim}

Le quatri�me test montre que l'impl�mentation fournie par 
le compilateur Intel donne un r�sultat correct, bien que la 
m�thode de Smith donne de mauvais r�sultats.

\subsubsection{C Code}

Le code C++ suivant illustre le traitement du probl�me.
Il est fond� sur le type "double complex" et fonctionne avec le compilateur 
Intel C 11.0 avec la configuration du standard C99 dans l'environnement Visual Studio.

\begin{verbatim}
#include <stdio.h> 
#include <math.h>
#include <complex.h>

// 
// naive --
//   Compute the complex division with a naive method.
//
void naive (double ar, double ai, double br, double bi, double * cr, double * ci)
{
	double den;
	den = br * br + bi * bi;
	*cr = (ar * br + ai * bi) / den;
	*ci = (ai * br - ar * bi) / den;
}

// 
// smith --
//   Compute the complex division with Smith's method.
//
void smith (double ar, double ai, double br, double bi, double * cr, double * ci)
{
	double den;
	double r;
	double abr;
	double abi;
	abr = fabs(br);
	abi = fabs(bi);
	if ( abr >= abi)
	{
		r = bi / br;
		den = br + r*bi;
		*cr = (ar + ai*r) / den;
		*ci = (ai - ar*r) / den;
	}
	else
	{
		r = br / bi;
		den = bi + r*br;
		*cr = (ar*r + ai) / den;
		*ci = (ai*r - ar) / den;
	}
}

// 
// compare --
//   Compare 3 methods for complex division:
//   * naive method
//   * Smith method
//   * C99 method
//
void compare (double ar, double ai, double br, double bi)
{
	double complex a;
	double complex b;
	double complex c;

	double cr;
	double ci;

	printf("****************\n");

	naive(ar, ai, br, bi, &cr, &ci);
	printf("Naif  --> c = %e +  %e * I\n" , cr ,  ci );

	smith(ar, ai, br, bi, &cr, &ci);
	printf("Smith --> c = %e +  %e * I\n" , cr ,  ci );

	a = ar + ai*I;
	b = br + bi*I;
	c = a / b;
	printf("C     --> c = %e +  %e * I\n" , creal(c) ,  cimag(c) );
}


int main(void)
{
	double ar;
	double ai;
	double br;
	double bi;


	// Check that naive implementation does not have a bug
	ar = 1;
	ai = 2;
	br = 3;
	bi = 4;
	compare (ar, ai, br, bi);

	// Check that naive implementation is not robust with respect to overflow
	ar = 1;
	ai = 1;
	br = 1;
	bi = 1e307;
	compare (ar, ai, br, bi);

	// Check that naive implementation is not robust with respect to underflow
	ar = 1;
	ai = 1;
	br = 1e-308;
	bi = 1e-308;
	compare (ar, ai, br, bi);

	// Check that Smith method is not robust in complicated cases
	ar = 1e307;
	ai = 1e-307;
	br = 1e205;
	bi = 1e-205;
	compare (ar, ai, br, bi);

	return 0;
}
\end{verbatim}

Voici le r�sultat qui appara�t dans la console :

\begin{verbatim}
****************
Naif  --> c = 4.400000e-001 +  8.000000e-002 * I
Smith --> c = 4.400000e-001 +  8.000000e-002 * I
C     --> c = 4.400000e-001 +  8.000000e-002 * I
****************
Naif  --> c = 0.000000e+000 +  -0.000000e+000 * I
Smith --> c = 1.000000e-307 +  -1.000000e-307 * I
C     --> c = 1.000000e-307 +  -1.000000e-307 * I
****************
Naif  --> c = 1.#INF00e+000 +  -1.#IND00e+000 * I
Smith --> c = 1.000000e+308 +  0.000000e+000 * I
C     --> c = 1.000000e+308 +  0.000000e+000 * I
****************
Naif  --> c = -1.#IND00e+000 +  -0.000000e+000 * I
Smith --> c = 1.000000e+102 +  0.000000e+000 * I
C     --> c = 1.000000e+102 +  -1.000000e-308 * I
\end{verbatim}

Cela montre que l'impl�mentation des nombres complexes dans la librairie 
fournie par Intel traite le probl�me de mani�re ad�quate.

\subsection{Explanations}
\subsubsection{The Smith's method}

C'est pourquoi les auteurs de Scilab, qui ont lu \cite{WhatEveryComputerScientist}, 
ont implement� la formule de Smith \cite{368661} (mais ils citent 
Goldberg en r�f�rence, ce qui est une erreur) :

\begin{algorithmic}
\IF {$( |d| <= |c| )$}
  \STATE $r \gets d / c$
  \STATE $den \gets c + r * d$
  \STATE $e \gets (a + b * r)/ den $
  \STATE $f \gets (b - a * r)/ den $
\ELSE
  \STATE $r \gets c / d$
  \STATE $den  \gets d + r * c$
  \STATE $e \gets (a * r + b)/den $
  \STATE $f \gets (b * r - a)/den$
\ENDIF
\end{algorithmic}

Dans le cas $(1+i)/(1+1e308 i)$, la m�thode de Smith donne 

\begin{verbatim}
si ( |1e308| <= |1| ) > test faux
sinon
  r = 1 / 1e308 = 0
  den  = 1e308 +  0 * 1 = 1e308
  e = (1 * 0 + 1) / 1e308 = 1e-308
  f = (1 * 0 - 1) / 1e308 = -1e-308
\end{verbatim}

ce qui est le r�sultat correct.

Dans le cas $(1+i)/(1e-308+1e-308 i)$, la m�thode de Smith donne 

\begin{verbatim}
si ( |1e-308| <= |1e-308| ) > test vrai
  r = 1e-308 / 1e308 = 1
  den  = 1e-308 +  1 * 1e-308 = 2e308
  e = (1 + 1 * 1) / 2e308 = 1e308
  f = (1 - 1 * 1) / 2e308 = 0
\end{verbatim}

ce qui est encore une fois le r�sultat correct.

Il s'av�re que le calcul de Smith, �crit en 1962, fonctionne bien dans un 
certain nombre de situations. L'article [4] cite une analyse de 
Hough qui donne une borne sur l'erreur r�alis�e par 
le calcul. 
\begin{verbatim}
|zcomp - zref| <= eps |zref|
\end{verbatim}

\subsubsection{The limits of the Smith method}

L'article \cite{214414} (1985) toutefois, fait la distinction 
entre la norme |zcomp - zref| et la valeur des 
parties imaginaires et r�elles. Il montre en particulier un 
exemple dans lequel la partie imaginaire est erron�e.

Supposons que m et n sont des entiers poss�dant les propri�t�s suivantes 
\begin{verbatim}
m >> 0
n >> 0
n >> m
\end{verbatim}

On peut alors facilement d�montrer que la division complexe 
suivante peut �tre approch�e :

\begin{verbatim}
10^n + i 10^-n
-------------- = 10^(n-m) - i 10^(n-3m)
10^m + i 10^-m
\end{verbatim}

On d�cide alors de choisir les nombres n et m inf�rieurs � 308 mais de telle sorte 
que 
\begin{verbatim}
n - 3 m = -308
\end{verbatim}

Par exemple le couple m=205, n=307 satisfait les �galit�s pr�c�dentes 
de telle sorte que 

\begin{verbatim}
10^307 + i 10^-307
------------------ = 10^102 - i 10^-308
10^205 + i 10^-205
\end{verbatim}

Il est facile de voir que ce dernier cas met en d�faut la 
formulation na�ve. 
Il est plus surprenant de constater que ce cas met �galement 
en d�faut la formule de Smith. En effet, les op�rations suivantes 
sont r�alis�es par la m�thode de Smith

\begin{verbatim}
si ( |1e-205| <= |1e205| ) > test vrai
  r = 1e-205 / 1e205 = 0
  den  = 1e205 +  0 * 1e-205 = 1e205
  e = (10^307 + 10^-307 * 0) / 1e205 = 1e102
  f = (10^-307 - 10^307 * 0) / 1e205 = 0
\end{verbatim}

On constate que la partie r�elle est exacte tandis que la 
partie imaginaire est fausse. On peut �galement v�rifier 
que le module du r�sultat est domin� par la partie r�elle de 
telle sorte que l'in�galit� |zcomp - zref| <= eps |zref| reste 
v�rifi�e.

Les limites de la m�thode de Smith ont �t�es lev�es dans \cite{214414}.
L'algorithme propos� est fond� sur une proposition qui d�montre 
que si n nombres x1...xn sont repr�sentables alors min(xi) * max(xi)
est �galement repr�sentable. L'impl�mentation de 
la division complexe tire parti de cette proposition pour r�aliser un 
calcul correct.

Il s'av�re que l'algorithme de Stewart est d�pass� par l'algorithme
de Li et Al \cite{567808}, mais �galement par celui de Kahan \cite{KAHAN1987}, 
qui, d'apr�s \cite{1039814}, est 
similaire � celui impl�ment� dans le standard C99.





\chapter{Conclusion}

That tool might be extended in future releases so that it provides the following features :
\begin{itemize}
\item Kelley restart based on simplex gradient [9],
\item C-based implementation (a prototype is provided in appendix B),
\item parallel implementation of the DIRECT algorithm,
\item implementation of the Hook-Jeeves and Multidimensional Search methods [9]
\item parallel implementation of the Nelder-Mead algorithm. See for example [21]. 
?This paper generalizes the widely used Nelder and Mead (Comput J 
7:308?313, 1965) simplex algorithm to parallel processors. Unlike most 
previous parallelization methods, which are based on parallelizing the 
tasks required to compute a specific objective function given a vector 
of parameters, our parallel simplex algorithm uses parallelization at 
the parameter level. Our parallel simplex algorithm assigns to each 
processor a separate vector of parameters corresponding to a point on a 
simplex. The processors then conduct the simplex search steps for an 
improved point, communicate the results, and a new simplex is formed. 
The advantage of this method is that our algorithm is generic and can be 
applied, without re-writing computer code, to any optimization problem 
which the non-parallel Nelder?Mead is applicable. The method is also 
easily scalable to any degree of parallelization up to the number of 
parameters. In a series of Monte Carlo experiments, we show that this 
parallel simplex method yields computational savings in some experiments 
up to three times the number of processors.?
\end{itemize}



\clearpage

\appendix

\section{Simple experiments}

In this section, we analyse the examples given in the introduction of this 
article.

\subsection{Why $0.1$ is rounded}

In this section, we present a brief explanation for the 
following Scilab session.

\begin{verbatim}
-->format(25)
-->x1=0.1
 x1  =
    0.1000000000000000055511  
-->x2 = 1.0-0.9
 x2  =
    0.0999999999999999777955  
-->x1==x2
 ans  =
  F  
\end{verbatim}

In fact, only the 17 first digits $0.100000000000000005$ are 
significant and the last digits are a artifact of Scilab's 
displaying system.

The number $0.1$ can be represented as the normalized number 
$1.0 \times 10^{-1}$. But the binary floating point representation
of $0.1$ is approximately \cite{WhatEveryComputerScientist} 
$1.100110011001100110011001... \times 2^{-4}$. As you see, the decimal
representation is made of a finite number of digits while the 
binary representation is made of an infinite sequence of 
digits. Because Scilab computations are based on double precision numbers
and because that numbers only have 64 bits to represent the number, 
some \emph{rounding} must be performed.

In our example, it happens that $0.1$ falls between two 
different binary floating point numbers. After rounding, 
the binary floating point number is associated with the decimal 
representation "0.100000000000000005", that is "rounding up" 
in this case. On the other side, $0.9$ is also not representable 
as an exact binary floating point number (but 1.0 is exactly represented). 
It happens that, after the substraction "1.0-0.9", the decimal representation of the 
result is "0.09999999999999997", which is different from the rounded 
value of $0.1$.

\subsection{Why $sin(\pi)$ is rounded}

In this section, we present a brief explanation of the following 
Scilab 5.1 session, where the function sinus is applied to the 
number $\pi$.

\begin{verbatim}
-->format(25)
-->sin(0.0)
 ans  =
    0.  
-->sin(%pi)
 ans  =
    0.0000000000000001224647  
\end{verbatim}

Two kinds of approximations are associated with the previous 
result
\begin{itemize}
\item $\pi=3.1415926...$ is approximated by Scilab 
as the value returned by $4*atan(1.0)$,
\item the $sin$ function is approximated by a polynomial.
\end{itemize}

This article is too short to make a complete presentation 
of the computation of elementary functions. The interested 
reader may consider the direct analysis of the Fdlibm library
as very instructive \cite{fdlibm}.
The "Elementary Functions" book by Muller \cite{261217}
is a complete reference on this subject.

In Scilab, the "sin" function is directly performed by a 
fortran source code (sci\_f\_sin.f) and no additionnal 
algorithm is performed directly by Scilab.
At the compiler level, though, the "sin" function is 
provided by a library which is compiler-dependent. 
The main structure of the algorithm which computes 
"in" is probably the following 

\begin{itemize}
\item scale the input $x$ so that in lies in a restricted
interval, 
\item use a polynomial approximation of the local 
behaviour of "sin" in the neighbourhood of 0, with a guaranteed
precision.
\end{itemize}

In the Fdlibm library for example, the scaling interval is 
$[-\pi/4,\pi/4]$. 
The polynomial approximation of the sine function has the general form

\begin{eqnarray}
sin(x) &\approx& x + a_3x^3 + \ldots + a_{2n+1} x^{2n+1}\\
&\approx & x + x^3 p(x^2)
\end{eqnarray}

In the Fdlibm library, 6 terms are used.

For the inverse tan "atan" function, which is 
used to compute an approximated value of $\pi$, the process is the same.
All these operations are guaranteed with some precision.
For example, suppose that the functions are guaranteed with 14 significant
digits. That means that 17-14 + 1 = 3 digits may be rounded in the process.
In our current example, the value of $sin(\pi)$ is approximated 
with 17 digits after the point as "0.00000000000000012". That means that
2 digits have been rounded. 

\subsection{One more step}

In fact, it is possible to reduce the number of 
significant digits of the sine function to as low as 0 significant digits.
The mathematical theory is $sin(2^n \pi) = 0$, but that is not true with
floating point numbers. In the following Scilab session, we 

\begin{verbatim}
-->for i = 1:5
-->k=10*i;
-->n = 2^k;
-->sin(n*%pi)
-->end
 ans  =
  - 0.0000000000001254038322  
 ans  =
  - 0.0000000001284135242063  
 ans  =
  - 0.0000001314954487872237  
 ans  =
  - 0.0001346513391512239052  
 ans  =
  - 0.1374464882277985633419  
\end{verbatim}

For $sin(2^{50})$, all significant digits are lost. This computation
may sound \emph{extreme}, but it must be noticed that it is inside the 
double precision possibility, since $2^{50} \approx 3.10^{15} \ll 10^{308}$.
The solution may be to use multiple precision numbers, such as in the 
Gnu Multiple Precision system.

If you know a better algorithm, based on double precision only, 
which allows to compute accurately such kind of values, the Scilab 
team will surely be interested to hear from you !



%\section{The Pythagorean sum}

In this section, we analyse the computation of the Pythagorean sum,
which is used in two different computations, that is the norm of a complex
number and the 2-norm of a vector of real values.

In the first part, we briefly present the mathematical formulas for these 
two computations.
We then present the na�ve algorithm based on these mathematical formulas. 
In the second part, we make some experiments in Scilab and compare our
na�ve algorithm with the \emph{abs} and \emph{norm} Scilab primitives.
In the third part, we analyse 
why and how floating point numbers must be taken into account when the 
Pythagorean sum is to compute.

\subsection{Theory}

\subsection{Experiments}

% TODO : compare both abs and norm.

\lstset{language=Scilab}
\lstset{numbers=left}
\lstset{basicstyle=\footnotesize}
\lstset{keywordstyle=\bfseries}
\begin{lstlisting}
// Straitforward implementation
function mn2 = mynorm2(a,b)
  mn2 = sqrt(a^2+b^2)
endfunction
// With scaling
function mn2 = mypythag1(a,b)
  if (a==0.0) then
    mn2 = abs(b);
  elseif (b==0.0) then
    mn2 = abs(a);
  else
    if (abs(b)>abs(a)) then
      r = a/b;
      t = abs(b);
    else
      r = b/a;
      t = abs(a);
    end
    mn2 = t * sqrt(1 + r^2);
  end
endfunction
// With Moler & Morrison's
// At most 7 iterations are required.
function mn2 = mypythag2(a,b)
  p = max(abs(a),abs(b))
  q = min(abs(a),abs(b))
  //index = 0
  while (q<>0.0)
    //index = index + 1
    //mprintf("index = %d, p = %e, q = %e\n",index,p,q)
    r = (q/p)^2
    s = r/(4+r)
    p = p + 2*s*p
    q = s * q
  end
  mn2 = p
endfunction
function compare(x)
  mprintf("Re(x)=%e, Im(x)=%e\n",real(x),imag(x));
  p = abs(x);
  mprintf("%20s : %e\n","Scilab",p);
  p = mynorm2(real(x),imag(x));
  mprintf("%20s : %e\n","Naive",p);
  p = mypythag1(real(x),imag(x));
  mprintf("%20s : %e\n","Scaling",p);
  p = mypythag2(real(x),imag(x));
  mprintf("%20s : %e\n","Moler & Morrison",p);
endfunction
// Test #1 : all is fine
x = 1 + 1 * %i;
compare(x);
// Test #2 : more difficult when x is large
x = 1.e200 + 1 * %i;
compare(x);
// Test #3 : more difficult when x is small
x = 1.e-200 + 1.e-200 * %i;
compare(x);
\end{lstlisting}

\begin{verbatim}
***************************************
Example #1 : simple computation with Scilab 5.1
x(1)=1.000000e+000, x(2)=1.000000e+000
              Scilab : 1.414214e+000
               Naive : 1.414214e+000
             Scaling : 1.414214e+000
    Moler & Morrison : 1.414214e+000
***************************************
Example #2 : with large numbers ?
              Scilab : Inf
               Naive : Inf
             Scaling : 1.000000e+200
    Moler & Morrison : 1.000000e+200
***************************************
Example #3 : with small numbers ?
x(1)=1.000000e-200, x(2)=1.000000e-200
              Scilab : 0.000000e+000
               Naive : 0.000000e+000
             Scaling : 1.414214e-200
    Moler & Morrison : 1.414214e-200
***************************************
> Conclusion : Scilab is naive !
Octave 3.0.3
***************************************
octave-3.0.3.exe:29> compare(x);
***************************************
x(1)=1.000000e+000, x(2)=1.000000e+000
              Octave : 1.414214e+000
               Naive : 1.414214e+000
             Scaling : 1.414214e+000
    Moler & Morrison : 1.414214e+000
***************************************
x(1)=1.000000e+200, x(2)=1.000000e+000
              Octave : 1.000000e+200
               Naive : Inf
             Scaling : 1.000000e+200
    Moler & Morrison : 1.000000e+200
***************************************
octave-3.0.3.exe:33> compare(x)
x(1)=1.000000e-200, x(2)=1.000000e-200
              Octave : 1.414214e-200
               Naive : 0.000000e+000
             Scaling : 1.414214e-200
    Moler & Morrison : 1.414214e-200
***************************************
> Conclusion : Octave is not naive !

With complex numbers.
***************************************

Re(x)=1.000000e+000, Im(x)=1.000000e+000
              Scilab : 1.414214e+000
               Naive : 1.414214e+000
             Scaling : 1.414214e+000
    Moler & Morrison : 1.414214e+000
***************************************
Re(x)=1.000000e+200, Im(x)=1.000000e+000
              Scilab : 1.000000e+200
               Naive : Inf
             Scaling : 1.000000e+200
    Moler & Morrison : 1.000000e+200
***************************************
Re(x)=1.000000e-200, Im(x)=1.000000e-200
              Scilab : 1.414214e-200
               Naive : 0.000000e+000
             Scaling : 1.414214e-200
    Moler & Morrison : 1.414214e-200
***************************************
> Conclusion : Scilab is not naive !
\end{verbatim}

\subsection{Explanations}

\subsection{References}

The paper by Moler and Morrisson 1983 \cite{journals/ibmrd/MolerM83} gives an 
algorithm to compute the Pythagorean sum $a\oplus b = \sqrt{a^2 + b^2}$
without computing their squares or their square roots. Their algorithm is based on a cubically
convergent sequence.
The BLAS linear algebra suite of routines \cite{900236} includes the SNRM2, DNRM2
and SCNRM2 routines which conpute the euclidian norm of a vector.
These routines are based on Blue \cite{355771} and Cody \cite{Cody:1971:SEF}.
In his 1978 paper \cite{355771}, James Blue gives an algorithm to compute the 
Euclidian norm of a n-vector $\|x\| = \sqrt{\sum_{i=1,n}x_i^2}$. 
The exceptionnal values of the \emph{hypot} operator are defined as the 
Pythagorean sum in the IEEE 754 standard \cite{P754:2008:ISF,ieee754-1985}.
The \emph{\_\_ieee754\_hypot(x,y)} C function is implemented in the 
Fdlibm software library \cite{fdlibm} developed by Sun Microsystems and 
available at netlib. This library is used by Matlab \cite{matlab-hypot}
and its \emph{hypot} command.






%% Bibliography


\addcontentsline{toc}{section}{Bibliography}
\bibliographystyle{plain}
\bibliography{scilabisnotnaive}


\end{document}

