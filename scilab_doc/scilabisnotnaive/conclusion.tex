\section{Conclusion}

We have presented several cases where the mathematically perfect 
algorithm (i.e. without obvious bugs) do not produce accurate results 
with the computer in particular situations.
In this paper, we have shown that specific methods can be used to 
cure some of the problems. We have also shown that these methods 
do not cure all the problems.

All Scilab algorithms take floating point values as inputs,
and returns floating point values as output. Problems arrive when the 
intermediate calculations involve terms which are 
not representable as floating point values.

That article should not discourage us 
from implementing our own algorithms. Rather, it should warn
us and that some specific work is to do when we translate the 
mathematical material into a algorithm.
That article shows us that accurate can be obtained with 
floating point numbers, provided that we are less \emph{na�ve}.

