\section{Complex division}

In that section, we analyse the problem of the complex division in Scilab.
We especially detail the difference between the mathematical, straitforward
formula and the floating point implementation. In the first part, we briefly report 
the formulas which allow to 
compute the real and imaginary parts of the division of two complex numbers.
We then present the na�ve algorithm based on these mathematical formulas. 
In the second part, we make some experiments in Scilab and compare our
na�ve algorithm with the \emph{/} Scilab operator.
In the third part, we analyse 
why and how floating point numbers must be taken into account when the 
implementation of such division is required.

\subsection{Theory}

The formula which allows to compute the 
real and imaginary parts of the division of two 
complex numbers is 
\begin{eqnarray}
\frac{a + ib}{c + id} = \frac{ac + bd}{c^2 + d^2} + i \frac{bc - ad}{c^2 + d^2} 
\end{eqnarray}

The naive algorithm for the computation of the complex division
is presented in figure \ref{naive-complexdivision}.

\begin{figure}[htbp]
\begin{algorithmic}
\STATE $den \gets c^2 + d^2$
\STATE $e \gets (ac + bd)/ den$
\STATE $f \gets (bc - ad)/ den$ 
\end{algorithmic}
\caption{Naive algorithm to compute the complex division}
\label{naive-complexdivision}
\end{figure}


\subsection{Experiments}

The following Scilab function is a straitforward implementation
of the previous formulas.

\lstset{language=Scilab}
\lstset{numbers=left}
\lstset{basicstyle=\footnotesize}
\lstset{keywordstyle=\bfseries}
\begin{lstlisting}
// 
// naive --
//   Compute the complex division with a naive method.
//
function [cr,ci] = naive (ar , ai , br , bi )
  den = br * br + bi * bi;
  cr = (ar * br + ai * bi) / den;
  ci = (ai * br - ar * bi) / den;
endfunction
\end{lstlisting}

In the following script, one compares the naive implementation
against the Scilab implementation with two cases.

\lstset{language=Scilab}
\lstset{numbers=left}
\lstset{basicstyle=\footnotesize}
\lstset{keywordstyle=\bfseries}
\begin{lstlisting}
  // Check that no obvious bug is in mathematical formula.
  [cr ci] = naive ( 1.0 , 2.0 , 3.0 , 4.0 )
  (1.0 + 2.0 * %i)/(3.0 + 4.0 * %i)
  // Check that mathematical formula does not perform well
  // when large number are used.
  [cr ci] = naive ( 1.0 , 1.0 , 1.0 , 1.e307 )
  (1.0 + 1.0 * %i)/(1.0 + 1.e307 * %i)
\end{lstlisting}

That prints out the following messages.

\begin{verbatim}
-->  // Check that no obvious bug is in mathematical formula.
-->  [cr ci] = naive ( 1.0 , 2.0 , 3.0 , 4.0 )
 ci  =
    0.08  
 cr  =
    0.44  
-->  (1.0 + 2.0 * %i)/(3.0 + 4.0 * %i)
 ans  =
    0.44 + 0.08i  
-->  // Check that mathematical formula does not perform well
-->  // when large number are used.
-->  [cr ci] = naive ( 1.0 , 1.0 , 1.0 , 1.e307 )
 ci  =
    0.  
 cr  =
    0.  
-->  (1.0 + 1.0 * %i)/(1.0 + 1.e307 * %i)
 ans  =
    1.000-307 - 1.000-307i  
\end{verbatim}

The simple calculation confirms that there is no bug in the 
naive implementation. But differences are apprearing when 
large numbers are used. In the second case, the naive 
implementation does not give a single exact digit !

To make more complete tests, the following script allows to 
compare the results of the naive and the Scilab methods.
We use three kinds of relative errors 
\begin{enumerate}
\item the relative error on the complex numbers, as a whole $e=\frac{|e - c|}{|e|}$,
\item the relative error on the real part $e=\frac{|e_r - e_r|}{e_r}$,
\item the relative error on the imaginary part $e=\frac{|e_i - e_i|}{e_i}$.
\end{enumerate}

\lstset{language=Scilab}
\lstset{numbers=left}
\lstset{basicstyle=\footnotesize}
\lstset{keywordstyle=\bfseries}
\begin{lstlisting}
// 
// compare --
//   Compare 3 methods for complex division:
//   * naive method
//   * Smith method
//   * C99 method
//
function compare (ar, ai, br, bi, rr, ri)
  printf("****************\n");
  printf("           a = %10.5e +  %10.5e * I\n" , ar ,  ai );
  printf("           b = %10.5e +  %10.5e * I\n" , br ,  bi );
  [cr ci] = naive ( ar, ai, br, bi);
  printf("Naive  --> c = %10.5e +  %10.5e * I\n" , cr ,  ci );
  c = cr + %i * ci
  r = rr + %i * ri;
  error1 = abs(r - c)/abs(r);
  if (rr==0.0) then
    error2 = abs(rr - cr);
  else
    error2 = abs(rr - cr)/abs(rr);
  end
  if (ri==0.0) then
    error3 = abs(ri - ci);
  else
    error3 = abs(ri - ci)/abs(ri);
  end
  printf("   e1 = %10.5e, e2 = %10.5e, e3 = %10.5e\n", error1, error2, error3);
  
  a = ar + ai * %i;
  b = br + bi * %i;
  c = a/b;
  cr = real(c);
  ci = imag(c);
  printf("Scilab --> c = %10.5e +  %10.5e * I\n" , cr ,  ci );
  c = cr + %i * ci
  error1 = abs(r - c)/abs(r);
  if (rr==0.0) then
    error2 = abs(rr - cr);
  else
    error2 = abs(rr - cr)/abs(rr);
  end
  if (ri==0.0) then
    error3 = abs(ri - ci);
  else
    error3 = abs(ri - ci)/abs(ri);
  end
  printf("   e1 = %10.5e, e2 = %10.5e, e3 = %10.5e\n", error1, error2, error3);
endfunction
\end{lstlisting}

In the following script, we compare the naive and the Scilab
implementations of the complex division with 4 couples of 
complex numbers. The first instruction "ieee(2)" configures the 
IEEE system so that Inf and Nan numbers are generated instead 
of Scilab error messages.

\lstset{language=Scilab}
\lstset{numbers=left}
\lstset{basicstyle=\footnotesize}
\lstset{keywordstyle=\bfseries}
\begin{lstlisting}
ieee(2);
// Check that naive implementation does not have a bug
ar = 1;
ai = 2;
br = 3;
bi = 4;
rr = 11/25;
ri = 2/25;
compare (ar, ai, br, bi, rr, ri);

// Check that naive implementation is not robust with respect to overflow
ar = 1;
ai = 1;
br = 1;
bi = 1e307;
rr = 1e-307;
ri = -1e-307;
compare (ar, ai, br, bi, rr, ri);

// Check that naive implementation is not robust with respect to underflow
ar = 1;
ai = 1;
br = 1e-308;
bi = 1e-308;
rr = 1e308;
ri = 0.0;
compare (ar, ai, br, bi, rr, ri);

\end{lstlisting}

The script then prints out the following messages.

\begin{verbatim}
****************
           a = 1.00000e+000 +  2.00000e+000 * I
           b = 3.00000e+000 +  4.00000e+000 * I
Naive  --> c = 4.40000e-001 +  8.00000e-002 * I
   e1 = 0.00000e+000, e2 = 0.00000e+000, e3 = 0.00000e+000
Scilab --> c = 4.40000e-001 +  8.00000e-002 * I
   e1 = 0.00000e+000, e2 = 0.00000e+000, e3 = 0.00000e+000
****************
           a = 1.00000e+000 +  1.00000e+000 * I
           b = 1.00000e+000 +  1.00000e+307 * I
Naive  --> c = 0.00000e+000 +  -0.00000e+000 * I
   e1 = 1.00000e+000, e2 = 1.00000e+000, e3 = 1.00000e+000
Scilab --> c = 1.00000e-307 +  -1.00000e-307 * I
   e1 = 2.09614e-016, e2 = 1.97626e-016, e3 = 1.97626e-016
****************
           a = 1.00000e+000 +  1.00000e+000 * I
           b = 1.00000e-308 +  1.00000e-308 * I
Naive  --> c = Inf +  Nan * I
   e1 = Nan, e2 = Inf, e3 = Nan
Scilab --> c = 1.00000e+308 +  0.00000e+000 * I
   e1 = 0.00000e+000, e2 = 0.00000e+000, e3 = 0.00000e+000
\end{verbatim}

The case \#2 and \#3 shows very surprising results.
With case \#2, the relative errors shows that the naive 
implementation does not give any correct digits.
In case \#3, the naive implementation produces Nan and Inf results.
In both cases, the Scilab command "/" gives accurate results, i.e.,
with at least 16 significant digits.

\subsection{Explanations}

In this section, we analyse the reason why the naive implementation
of the complex division leads to unaccurate results.
In the first section, we perform algebraic computations 
and shows the problems of the naive formulas.
In the second section, we present the Smith's method.

\subsubsection{Algebraic computations}

Let's analyse the second test and check the division of test \#2 :

\begin{eqnarray}
\frac{1 + I}{1 + 10^{307} I } = 10^{307} - I * 10^{-307}
\end{eqnarray}

The naive formulas leads to the following results.

\begin{eqnarray}
den &=& c^2 + d^2 = 1^2 + (10^{307})^2 = 1 + 10^{614} \approx 10^{614} \\
e &=& (ac + bd)/ den = (1*1 + 1*10^{307})/1e614 \approx 10^{307}/10^{614} \approx 10^{-307}\\
f &=& (bc - ad)/ den = (1*1 - 1*10^{307})/1e614 \approx -10^{307}/10^{614} \approx -10^{-307}
\end{eqnarray}

To understand what happens with the naive implementation, one should
focus on the intermediate numbers.
If one uses the naive formula with double precision numbers, then

\begin{eqnarray}
den = c^2 + d^2 = 1^2 + (10^{307})^2 = Inf
\end{eqnarray}

This generates an overflow, because $(10^{307})^2 = 10^{614}$ is not representable
as a double precision number.

The $e$ and $f$ terms are then computed as 

\begin{eqnarray}
e = (ac + bd)/ den = (1*1 + 1*10^{307})/Inf = 10^{307}/Inf = 0\\
f = (bc - ad)/ den = (1*1 - 1*10^{307})/Inf = -10^{307}/Inf = 0
\end{eqnarray}

The result is then computed without any single correct digit,
even though the initial numbers are all representable as double precision
numbers.

Let us check that the case \#3 is associated with an underflow.
We want to compute the following complex division :

\begin{eqnarray}
\frac{1 + I}{10^{-308} +  10^{-308} I}= 10^{308}
\end{eqnarray}

The naive mathematical formula gives 

\begin{eqnarray}
den &=& c^2 + d^2 = (10^{-308})^2 + (10^{-308})^2 = 10^{-616}10^{-616} + 10^{-616} = 2 \times 10^{-616} \\
e &=& (ac + bd)/ den = (1*10^{-308} + 1*10^{-308})/(2 \times 10^{-616}) \\
 &\approx&  (2 \times 10^{-308})/(2 \times 10^{-616}) \approx 10^{-308} \\
f &=& (bc - ad)/ den = (1*10^{-308} - 1*10^{-308})/(2 \times 10^{-616}) \approx 0/10^{-616} \approx 0
\end{eqnarray}

With double precision numbers, the computation is not performed this way.
Terms which are lower than $10^{-308}$ are too small to be representable 
in double precision and will be reduced to 0 so that an underflow occurs.

\begin{eqnarray}
den &=& c^2 + d^2 = (10^{-308})^2 + (10^{-308})^2 = 10^{-616} + 10^{-616} = 0 \\
e &=& (ac + bd)/ den = (1*10^{-308} + 1*10^{-308})/0 \approx 2\times 10^{-308}/0 \approx Inf \\
f &=& (bc - ad)/ den = (1*10^{-308} - 1*10^{-308})/0 \approx 0/0 \approx NaN \\
\end{eqnarray}

\subsubsection{The Smith's method}

In this section, we analyse the Smith's method and present the detailed
steps of this algorithm in the cases \#2 and \#3.

In Scilab, the algorithm which allows to perform the complex 
division is done by the the \emph{wwdiv} routine, which implements the 
Smith's method \cite{368661}, \cite{WhatEveryComputerScientist}.
The Smith's algorithm is based on normalization, which allow to 
perform the division even if the terms are large. 

The starting point of the method is the mathematical definition 

\begin{eqnarray}
\frac{a + ib}{c + id} = \frac{ac + bd}{c^2 + d^2} + i \frac{bc - ad}{c^2 + d^2} 
\end{eqnarray}

The method of Smith is based on the rewriting of this formula in 
two different, but mathematically equivalent formulas. The basic 
trick is to make the terms $d/c$ or $c/d$ appear in the formulas.
When $c$ is larger than $d$, the formula involving $d/c$ is used.
Instead, when $d$ is larger than $c$, the formula involving $c/d$ is 
used. That way, the intermediate terms in the computations rarely 
exceeds the overflow limits.

Indeed, the complex division formula can be written as 
\begin{eqnarray}
\frac{a + ib}{c + id} = \frac{a + b(d/c)}{c + d(d/c)} + i \frac{b - a(d/c)}{c + d(d/c)} \\
\frac{a + ib}{c + id} = \frac{a(c/d) + b}{c(d/c) + d} + i \frac{b(c/d) - a}{c(d/c) + d} 
\end{eqnarray}

These formulas can be simplified as 

\begin{eqnarray}
\frac{a + ib}{c + id} = \frac{a + br}{c + dr} + i \frac{b - ar}{c + dr}, \qquad r = d/c \\
\frac{a + ib}{c + id} = \frac{ar + b}{cr + d} + i \frac{br - a}{cr + d} , \qquad r = c/d
\end{eqnarray}

The Smith's method is based on the following algorithm.

\begin{algorithmic}
\IF {$( |d| <= |c| )$}
  \STATE $r \gets d / c$
  \STATE $den \gets c + r * d$
  \STATE $e \gets (a + b * r)/ den $
  \STATE $f \gets (b - a * r)/ den $
\ELSE
  \STATE $r \gets c / d$
  \STATE $den  \gets d + r * c$
  \STATE $e \gets (a * r + b)/den $
  \STATE $f \gets (b * r - a)/den$
\ENDIF
\end{algorithmic}

As we are going to check immediately, the Smith's method 
performs very well in cases \#2 and \#3.

In the case \#2 $\frac{1+i}{1+10^{-308} i}$, the Smith's method is

\begin{verbatim}
If ( |1e308| <= |1| ) > test false
Else
  r = 1 / 1e308 = 0
  den  = 1e308 +  0 * 1 = 1e308
  e = (1 * 0 + 1) / 1e308 = 
  f = (1 * 0 - 1) / 1e308 = -1e-308
\end{verbatim}

In the case \#3 $\frac{1+i}{10^{-308}+10^{-308} i}$, the Smith's method is 

\begin{verbatim}
If ( |1e-308| <= |1e-308| ) > test true
  r = 1e-308 / 1e308 = 1
  den  = 1e-308 +  1 * 1e-308 = 2e308
  e = (1 + 1 * 1) / 2e308 = 1e308
  f = (1 - 1 * 1) / 2e308 = 0
\end{verbatim}

\subsection{One more step}

In that section, we show the limitations of the Smith's method.

Suppose that we want to perform the following division 

\begin{eqnarray}
\frac{10^{307} + i 10^{-307}}{10^205 + i 10^{-205}} = 10^102 - i 10^-308
\end{eqnarray}

The following Scilab script allows to compare the naive implementation
and Scilab's implementation based on Smith's method.

\lstset{language=Scilab}
\lstset{numbers=left}
\lstset{basicstyle=\footnotesize}
\lstset{keywordstyle=\bfseries}
\begin{lstlisting}
// Check that Smith method is not robust in complicated cases
ar = 1e307;
ai = 1e-307;
br = 1e205;
bi = 1e-205;
rr = 1e102;
ri = -1e-308;
compare (ar, ai, br, bi, rr, ri);
\end{lstlisting}

When executed, the script produces the following output.

\begin{verbatim}
****************
           a = 1.00000e+307 +  1.00000e-307 * I
           b = 1.00000e+205 +  1.00000e-205 * I
Naive  --> c = Nan +  -0.00000e+000 * I
   e1 = 0.00000e+000, e2 = Nan, e3 = 1.00000e+000
Scilab --> c = 1.00000e+102 +  0.00000e+000 * I
   e1 = 0.00000e+000, e2 = 0.00000e+000, e3 = 1.00000e+000
\end{verbatim}

As expected, the naive method produces a Nan.
More surprisingly, the Scilab output is also quite approximated.
More specifically, the imaginary part is computed as zero, although
we know that the exact result is $10^-308$, which is representable 
as a double precision number.
The relative error based on the norm of the complex number is 
accurate ($e1=0.0$), but the relative error based on the imaginary
part only is wrong ($e3=1.0$), without any correct digits.

The reference \cite{1667289} cites an analysis by Hough which 
gives a bound for the relative error produced by the Smith's method

\begin{eqnarray}
|zcomp - zref| <= eps |zref|
\end{eqnarray}

The paper \cite{214414} (1985), though, makes a distinction between
the norm $|zcomp - zref|$ and the relative error for the 
real and imaginary parts. It especially gives an example
where the imaginary part is wrong.

In the following paragraphs, we detail the derivation
of an example inspired by \cite{214414}, but which 
shows the problem with double precision numbers (the example
in \cite{214414} is based on an abstract machine with 
exponent range $\pm 99$).

Suppose that $m,n$ are integers so that the following 
conditions are satisfied
\begin{eqnarray}
m >> 0\\
n >> 0\\
n >> m
\end{eqnarray}

One can easily proove that the complex division can be approximated 
as  
\begin{eqnarray}
\frac{10^n + i 10^{-n}}{10^m + i 10^{-m}} &=& 
\frac{10^{n+m} + 10^{-(m+n)}}{10^{2m} + 10^{-2m}} +  
i \frac{10^{m-n} - 10^{n-m}}{10^{2m} + 10^{-2m}}
\end{eqnarray}

Because of the above assumptions, that leads to the following 
approximation 
\begin{eqnarray}
\frac{10^n + i 10^{-n}}{10^m + i 10^{-m}} \approx 10^{n-m} - i 10^{n-3m}
\end{eqnarray}
which is correct up to approximately several 100 digits.

One then consider $m,n<308$ but so that 
\begin{eqnarray}
n - 3 m = -308
\end{eqnarray}

For example, the couple $m=205$, $n=307$ satisfies all conditions.
That leads to the complex division 

\begin{eqnarray}
\frac{10^{307} + i 10^{-307}}{10^{205} + i 10^{-205}} = 10^{102} - i 10^{-308}
\end{eqnarray}

It is easy to check that the naive implementation does not 
proove to be accurate on that example.
We have already shown that the Smith's method is failing to 
produce a non zero imaginary part. Indeed, the steps of the 
Smith algorithm are the following 

\begin{verbatim}
If ( |1e-205| <= |1e205| ) > test true
  r = 1e-205 / 1e205 = 0
  den  = 1e205 +  0 * 1e-205 = 1e205
  e = (10^307 + 10^-307 * 0) / 1e205 = 1e102
  f = (10^-307 - 10^307 * 0) / 1e205 = 0
\end{verbatim}

The real part is accurate, but the imaginary part has no 
correct digit. One can also check that the inequality $|zcomp - zref| <= eps |zref|$ 
is still true.

The limits of Smith's method have been reduced in Stewart's paper \cite{214414}.
The new algorithm is based on the theorem which states that if $x_1 \ldots x_n$
are $n$ floating point representable numbers then $\min_{i=1,n}(x_i) \max_{i=1,n}(x_i)$
is also representable. The algorithm uses that theorem to perform a 
correct computation.

Stewart's algorithm is superseded by the one by Li et Al \cite{567808}, but 
also by Kahan's \cite{KAHAN1987}, which, from  \cite{1039814}, is the one implemented
in the C99 standard.

\subsection{References}

The 1962 paper by R. Smith \cite{368661} describes the algorithm which is used in 
Scilab. The Goldberg paper \cite{WhatEveryComputerScientist} introduces many 
of the subjects presented in this document, including the problem of the 
complex division. The 1985 paper by Stewart \cite{214414} gives insight to 
distinguish between the relative error of the complex numbers and the relative
error made on real and imaginary parts. It also gives an algorithm based 
on min and max functions. Knuth's bible \cite{artcomputerKnuthVol2} presents
the Smith's method in section 4.2.1, as exercize 16. Knuth gives also 
references \cite{Wynn:1962:AAP} and \cite{DBLP:journals/cacm/Friedland67}.
The 1967 paper by Friedland \cite{DBLP:journals/cacm/Friedland67} describes 
two algorithm to compute the absolute value of a complex number 
$|x+iy| = \sqrt{x^2+y^2}$ and the square root of a 
complex number $\sqrt{x+iy}$.



