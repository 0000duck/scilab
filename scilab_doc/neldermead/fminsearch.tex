\chapter{The \scifunction{fminsearch} function}

In this chapter, we analyze the implementation of the \scifunction{fminsearch}
which is provided in Scilab. In the first part, we describe the specific 
choices of this implementation with respect to the Nelder-Mead algorithm.
In the second part, we present some numerical experiments which 
allows to check that the feature is behaving as expected, by comparison 
to Matlab's \scifunction{fminsearch}.

\section{\scifunction{fminsearch}'s algorithm}

\subsection{The algorithm}

The algorithm used is the Nelder-Mead algorithm. This corresponds to the 
"variable" value of the "-method" option of the \scifunction{neldermead}.
The "non greedy" version is used, that is, the expansion point is 
accepted only if it improves over the reflection point.

\subsection{The initial simplex}

The fminsearch algorithm uses a special initial simplex, which is an 
heuristic depending on the initial guess. The strategy chosen by 
fminsearch corresponds to the -simplex0method flag of the neldermead 
component, with the "pfeffer" method. It is associated with the -
simplex0deltausual = 0.05 and -simplex0deltazero = 0.0075 parameters. 
Pfeffer's method is an heuristic which is presented in "Global 
Optimization Of Lennard-Jones Atomic Clusters" by Ellen Fan \cite{Fan2002}. 
It is due to L. Pfeffer at Stanford. See in the help of optimsimplex for more 
details.

\subsection{The number of iterations}

In this section, we present the default values for the number of 
iterations in fminsearch.

The options input argument is an optionnal data structure which can 
contain the options.MaxIter field. It stores the maximum number of 
iterations. The default value is 200n, where n is the number of 
variables. The factor 200 has not been chosen by chance, but is the 
result of experiments performed against quadratic functions with 
increasing space dimension.

This result is presented in "Effect of dimensionality on the nelder-mead 
simplex method" by Lixing Han and Michael Neumann \cite{HanNeumann2006}. This paper is based 
on Lixing Han's PhD, "Algorithms in Unconstrained Optimization" \cite{Han2000}. The 
study is based on numerical experiment with a quadratic function where 
the number of terms depends on the dimension of the space (i.e. the 
number of variables). Their study shows that the number of iterations 
required to reach the tolerance criteria is roughly 100n. Most 
iterations are based on inside contractions. Since each step of the 
Nelder-Mead algorithm only require one or two function evaluations, the 
number of required function evaluations in this experiment is also 
roughly 100n.

\subsection{The termination criteria}

The algorithm used by \scifunction{fminsearch} uses a particular 
termination criteria, based both on the absolute size of the 
simplex and the difference of the function values in the simplex.
This termination criteria corresponds to the "-tolssizedeltafvmethod"
termination criteria of the \scifunction{neldermead} component.

The size of the simplex is computed with the $\sigma-+$ method,
which corresponds to the "sigmaplus" method of the \scifunction{optimsimplex}
component. The tolerance associated with this criteria is 
given by the "TolX" parameter of the \scifunction{options} data structure.
Its default value is 1.e-4.

The function value difference is the difference 
between the highest and the lowest function value in the simplex.
The tolerance associated with this criteria is given by the 
"TolFun" parameter of the \scifunction{options} data structure.
Its default value is 1.e-4.

\section{Numerical experiments}

In this section, we analyse the behaviour of Scilab's \scifunction{fminsearch}
function, by comparison of Matlab's \scifunction{fminsearch}. We especially analyse 
the results of the optimization, so that we can check that the algorithm
is indeed behaving the same way, even if the implementation is completely 
different. Our test case is based on Rosenbrock's function.

The following Matlab script allows to see the behaviour of Matlab's \scifunction{fminsearch}
function on Rosenbrock's test case.

\lstset{language=scilabscript}
\begin{lstlisting}
format long
banana = @(x)100*(x(2)-x(1)^2)^2+(1-x(1))^2;
[x,fval,exitflag,output] = fminsearch(banana,[-1.2, 1])
output.message
\end{lstlisting}

When this script is launched in Matlab, the following output is 
produced.

\lstset{language=scilabscript}
\begin{lstlisting}
>> format long
>> banana = @(x)100*(x(2)-x(1)^2)^2+(1-x(1))^2;
>> [x,fval] = fminsearch(banana,[-1.2, 1])
>> [x,fval,exitflag,output] = fminsearch(banana,[-1.2, 1])
x =
   1.000022021783570   1.000042219751772
fval =
     8.177661197416674e-10
exitflag =
     1
output =
    iterations: 85
     funcCount: 159
     algorithm: 'Nelder-Mead simplex direct search'
       message: [1x194 char]
>> output.message
ans =
Optimization terminated:
the current x satisfies the termination criteria using
OPTIONS.TolX of 1.000000e-04
and F(X) satisfies the convergence criteria using
OPTIONS.TolFun of 1.000000e-04
\end{lstlisting}

The following Scilab script allows to solve the problem with Scilab's 
\scifunction{fminsearch}.

\lstset{language=scilabscript}
\begin{lstlisting}
format(25)
function y = banana (x)
  y = 100*(x(2)-x(1)^2)^2 + (1-x(1))^2;
endfunction
[x , fval , exitflag , output] = fminsearch ( banana , [-1.2 1] )
output.message
\end{lstlisting}

The output associated with this Scilab script is the following.

\lstset{language=scilabscript}
\begin{lstlisting}
-->format(25)
-->function y = banana (x)
-->  y = 100*(x(2)-x(1)^2)^2 + (1-x(1))^2;
-->endfunction
-->[x , fval , exitflag , output] = fminsearch ( banana , [-1.2 1] )
 output  =
   algorithm: "Nelder-Mead simplex direct search"
   funcCount: 159
   iterations: 85
   message: [3x1 string]
 exitflag  =
    1.  
 fval  =
    0.0000000008177661099387  
 x  =
    1.0000220217835567027009    1.0000422197517710998227  
-->output.message
 ans  =
 
!Optimization terminated:                                                              !
!                                                                                      !
!the current x satisfies the termination criteria using OPTIONS.TolX of 1.000000e-004  !
!                                                                                      !
!and F(X) satisfies the convergence criteria using OPTIONS.TolFun of 1.000000e-004     !
\end{lstlisting}

The result is as close as possible to the result produced 
by Matlab. More specifically :
\begin{itemize}
\item the optimum $x$ is the same up to 14 significant digits,
\item the function value at optimum is the same up to 9 significant digits,
\item the number of iterations is the same,
\item the number of function evaluations is the same,
\item the exit flag is the same,
\item the content of the output is the same (but the string is not 
display the same way).
\end{itemize}


%TODO : check that the iterations are the same

