\chapter{Simplex theory}

In this section, we present the various definitions connected
to simplex algorithms. We introduce several methods to measure 
the size of a simplex, including the oriented length. 
We present several methods to compute an
initial simplex, for example the regular simplex used by Spendley et al..
We also present the simplex gradient, which is a forward or a centered 
difference formula for the gradient of the cost function.
The core of this section is from \cite{Kelley1999}.

\section{The simplex}

A \emph{simplex} $S$ in $\RR^n$ is the convex hull of $n+1$ points $S=\{\bx_i\}_{i=1,n+1}$.

Box extended the Nelder-Mead algorithm to handle bound and non linear constraints \cite{Box1965}.
To be able to manage difficult cases, he uses a \emph{complex} made of $k\geq n+1$ vertices.
In this section, we will state clearly when the definition and results can be applied to a complex. 
Indeed, some definitions such as the simplex gradient cannot be extended to a \emph{complex}
and are only applicable to a \emph{simplex}.

The point $\bx_i\in\RR^n$ is the $i$-th vertex of $S$. Given a function $f(\bx)\in\RR$,
each vertex is associated with a function value $f_i = f(\bx_i)$ for $i=1,n+1$.
In simplex algorithms, the vertex are sorted by increasing function values 

\begin{eqnarray}
\label{simplex-sortedfv}
f_1 \leq f_2 \leq \ldots \leq f_n \leq f_{n+1}
\end{eqnarray}

The sorting order is not precisely defined neither in Spendley's et al paper \cite{Spendley1962}
nor in Nelder and Mead's \cite{citeulike:3009487}. 
In \cite{lagarias:112}, the sorting rules are defined precisely to be able to 
state a theoretical convergence result. In practical implementations, though, the 
ordering rules have no measurable influence.

Let $V$ denote the $n\times n$ matrix of simplex directions 

\begin{eqnarray}
\label{simplex-directions}
V(S) = (\bx_2 - \bx_1, \bx_3 - \bx_1 , \ldots , \bx_{n+1} - \bx_1) = (\bv_1, \ldots , \bv_n)
\end{eqnarray}

We say that $S$ is nonsingular if the matrix of simplex directions $V(S)$ is nonsingular.


\section{The size of the simplex}

Several methods are available to compute the size of a simplex.
In Kelley's book \cite{Kelley1999}, the author presents the diameter and the two oriented lengths.

The simplex diameter $diam(S)$ is defined by 

\begin{eqnarray}
\label{simplex-diameter}
diam(S) = \max_{i,j=1,n+1} \|\bx_i - \bx_j\|
\end{eqnarray}

In practical implementations, computing the diameter requires two nested loops over the 
vertices of the simplex, i.e. $(n+1)^2$ operations. This is why authors generally 
prefer to use lengths which are less expensive to compute.

The two oriented lengths $\sigma_-(S)$ and $\sigma_+(S)$ are using the 
first vertex as the reference point and are defined by 

\begin{eqnarray}
\label{simplex-sigma}
\sigma_+(S) = \max_{i=2,n+1} \|\bx_i - \bx_1\| \qquad \textrm { and } \qquad \sigma_-(S) = \min_{i=2,n+1} \|\bx_i - \bx_1\|
\end{eqnarray}

The following inequalities are satisfied between the diameter and the maximum oriented length

\begin{eqnarray}
\label{simplex-sigma-diam}
\sigma_+(S) \leq diam(S) \leq 2 \sigma_+(S)
\end{eqnarray}

In Nash's book \cite{nla.cat-vn1060620}, the size of the simplex $s_N(S)$ is measured
based on the $\l-1$ norm and is defined by 

\begin{eqnarray}
\label{simplex-sizenash}
s_N(S) = \sum_{i=2,n+1} \|\bx_i - \bx_1\|
\end{eqnarray}

where 

\begin{eqnarray}
\label{simplex-sizenash2}
\|\bx_i - \bx_1\| = \sum_{i=1,n+1} |x_i^j - x_1^j|
\end{eqnarray}

where $x_i^j\in\RR$ is the $j$-th coordinate of the $i$-th vertex of the simplex $S$.

\section{The initial simplex}

TODO...

\section{The simplex gradient}

TODO...

