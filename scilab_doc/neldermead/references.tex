\section{References}

In the section 4.1.1 "Starting Points" of the \cite{} by ,
one can find an algorithm to initialize the initial simplex.

\lstset{language=Scilab}
\lstset{numbers=left}
\begin{lstlisting}
// 1. Shifting the unit simplex of dimension n from the origin to xin.
u = eye(n);
v(:,1) = xin;
for j=1:n
  y = u(:,j)+x;
  v(:,j+1) = y;
  x(:) = y;
  f = feval(funfcn,x,varargin{:});
  fv(1,j+1) = f;
end
\end{lstlisting}


"The second method creates a simplex with specified edge length
and orientation that depends on the given starting point of the coordinate
directions. If it is in the same direction as an optimal solution, this initial
simplex may push the process fast towards the optimum. The risk is that
the constructed simplex may be very flat. In other words, the shape of the
initial simplex is constructed with sharp angles. If the coordinate direction
of the given starting point is orthogonal to the direction towards an optimal
solution, then it will take more iterations to find an optimal solution
or, in the worst case, fail to find an optimal solution at all. The parameters
usual_delta and zero_term_delta are used to adjust the simplex orientation
and can be modified as needed."

\lstset{language=Scilab}
\lstset{numbers=left}
\begin{lstlisting}
// 2. Using a modified point set suggested by L. Pfeffer at Stanford [MAT]
usual_delta = 0.05; 
zero_term_delta = 0.0075;
for j = 1:n
  y = xin;
  if y(j) ~= 0
    y(j) = (1 + usual_delta)*y(j);
  else
    y(j) = zero_term_delta;
  end
  v(:,j+1) = y;
  x(:) = y; 
  f = feval(funfcn,x,varargin{:});
  fv(1,j+1) = f;
end
\end{lstlisting}

