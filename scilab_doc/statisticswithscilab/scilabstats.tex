%
% scilabstats.tex --
%   Some notes about Scilab statistical features in Scilab.
%
% Copyright 2008 Michael Baudin
%
\documentclass[12pt]{report}

%% Good fonts for PDF
\usepackage[cyr]{aeguill}

%% Package for page headers
\usepackage{fancyhdr}

%% Package to include graphics
%% Comment for DVI
\usepackage[pdftex]{graphicx}

%% Figures formats: jpeg or pdf
%% Comment for DVI
\DeclareGraphicsExtensions{.jpg,.pdf}

%% Package to create Hyperdocuments
%% Comment for DVI
\usepackage[pdftex,colorlinks=true,linkcolor=blue,citecolor=blue,urlcolor=blue]{hyperref}

%% Package to control printed area size
\usepackage{anysize}
%% ...by defining margins {left}{right}{top}{bottom}
\marginsize{22mm}{14mm}{12mm}{25mm}

%% Package used to include a bibliography
\usepackage{natbib}

%% R for real numbers
\usepackage{amssymb}

%% User defined commands

\usepackage{url}

% Scilab macros
\newcommand{\scimacro}[1]{\textit{#1}}
\newcommand{\scicommand}[1]{\textit{#1}}

% To highlight source code
\usepackage{listings}

% Define theorem environments 
\newtheorem{theorem}{Theorem}[section]
\newtheorem{lemma}[theorem]{Lemma}
\newtheorem{proposition}[theorem]{Proposition}
\newtheorem{corollary}[theorem]{Corollary}

\newenvironment{proof}[1][Proof]{\begin{trivlist}
\item[\hskip \labelsep {\bfseries #1}]}{\end{trivlist}}
\newenvironment{definition}[1][Definition]{\begin{trivlist}
\item[\hskip \labelsep {\bfseries #1}]}{\end{trivlist}}
\newenvironment{example}[1][Example]{\begin{trivlist}
\item[\hskip \labelsep {\bfseries #1}]}{\end{trivlist}}
\newenvironment{remark}[1][Remark]{\begin{trivlist}
\item[\hskip \labelsep {\bfseries #1}]}{\end{trivlist}}

\newcommand{\qed}{\nobreak \ifvmode \relax \else
      \ifdim\lastskip<1.5em \hskip-\lastskip
      \hskip1.5em plus0em minus0.5em \fi \nobreak
      \vrule height0.75em width0.5em depth0.25em\fi}

% Maths shortcuts 
\newcommand{\RR}{\mathbb{R}}

% Algorithms
\usepackage{algorithm2e}

\begin{document}
\author{Michael Baudin}
\date{February 2009}
\title{Scilab and Statistics}
\begin{abstract}
In this document, we describe the statistical features of Scilab.
We analyse the features available in Scilab's core (i.e. provided
"out of the box") and Scilab Statistical Toolboxes.
For Scilab's core statistical features, we analyse the different
libraries used by Scilab and provide a complete overview of 
the functions. For the most important features, we present Scilab 
sessions with a sample use of the command. Several Scilab Toolboxes
are analysed in this document, including Sci\_R and Stixbox.
We also analyse the missing features (not provided in the core and not in the 
toolboxes) with the tools which are available in other languages,
including Matlab and R.
\end{abstract}

\maketitle

\tableofcontents

\chapter{Introduction}

The Nelder-Mead simplex algorithm, published in 1965, is an enormously 
popular search method for multidimensional unconstrained optimization. 
The Nelder-Mead algorithm should not be confused with the (probably) 
more famous simplex algorithm of Dantzig for linear programming. The 
Nelder-Mead algorithm is especially popular in the fields of chemistry, 
chemical engineering, and medicine. Two measures of the ubiquity of the 
Nelder-Mead algorithm are that it appears in the best-selling handbook 
Numerical Recipes and in Matlab. In \cite{Torczon89multi-directionalsearch},
Virginia Torczon writes : "Margaret Wright has stated that over
fifty percent of the calls received by the support group for the NAG
software library concerned the version of the Nelder-Mead 
simplex algorithm to be found in that library". No derivative of the cost function is 
required, which makes the algorithm interesting for noisy problems.

The Nelder-Mead algorithm falls in the more general class of direct 
search algorithms. These methods use values of $f$ taken from a set of 
sample points and use that information to continue the sampling. The 
Nelder-Mead algorithm maintains a simplex which are approximations of an 
optimal point. The vertices are sorted according to the objective 
function values. The algorithm attemps to replace the worst vertex with 
a new point, which depends on the worst point and the centre of the best 
vertices. 

The goal of this toolbox is to provide a Nelder-Mead direct search optimization method to solve the 
following unconstrained optimization problem
\begin{eqnarray}
\min f(x)
\end{eqnarray}
where $x\in \RR^n$ where $n$ is the number of optimization parameters.
The Box complex algorithm, which is an extension of Nelder-Mead's algorithm, solves the 
following constrained problem
\begin{eqnarray}
&&\min f(x)\\
&&\ell_i \leq x_i \leq u_i, \qquad i = 1,n\\
&&g_j(x)\geq 0, \qquad j = 1, m\\
\end{eqnarray}
where $m$ is the number of nonlinear, positive constraints and $\ell_i,u_i\in \RR^n$ are the lower 
and upper bounds of the variables.

The Nelder-Mead algorithm may be used in the following optimization context :
\begin{itemize}
\item there is no need to provide the derivatives of the objective function,
\item the number of parameters is small (up to 10-20),
\item there are bounds and/or non linear constraints.
\end{itemize}

The internal design of the system is based on the following components :
\begin{itemize}
\item "neldermead" provides various Nelder-Mead variants and manages for Nelder-Mead specific settings, such as the method to compute the initial simplex, the specific termination criteria,
\item "fminsearch" provides a Scilab commands which aims at behaving as Matlab's fminsearch. Specific terminations criteria, initial simplex and auxiliary settings are automatically configured so that the behaviour of Matlab's fminsearch is exactly reproduced.
\item "optimset", "optimget" provides Scilab commands to emulate their Matlab counterparts.
\item "nmplot" provides a high-level component which provides directly output pictures for Nelder-Mead algorithm.
\end{itemize}
The current toolbox is based on (and therefore requires) the following toolboxes 
\begin{itemize}
\item "optimbase" provides an abstract class for a general optimization component, including the number of variables, the minimum and maximum bounds, the number of non linear inequality constraints, the loggin system, various termination criteria, the cost function, etc...
\item "optimsimplex" provides a class to manage a simplex made of an arbitrary number of vertices, including the computation of a simplex by various methods (axes, regular, Pfeffer's, randomized bounds), the computation of the size by various methods (diameter, sigma +, sigma-, etc...),
\end{itemize}

The following is a list of features the Nelder-Mead prototype algorithm currently provides :
\begin{itemize}
\item Manage various simplex initializations
  \begin{itemize}
  \item initial simplex given by user,
  \item initial simplex computed with a length and along the coordinate axes,
  \item initial regular simplex computed with Spendley et al. formula
  \item initial simplex computed by a small perturbation around the initial guess point
  \end{itemize}
\item Manage cost function
  \begin{itemize}
  \item optionnal additionnal argument
  \item direct communication of the task to perform : cost function or inequality constraints
  \end{itemize}
\item Manage various termination criteria, including maximum number of iterations, tolerance on function value (relative or absolute), 
  \begin{itemize}
  \item tolerance on x (relative or absolute),
  \item tolerance on standard deviation of function value (original termination criteria in [3]),
  \item maximum number of evaluations of cost function,
  \item absolute or relative simplex size,
  \end{itemize}
\item Manage the history of the convergence, including
  \begin{itemize}
  \item history of function values,
  \item history of optimum point,
  \item history of simplices,
  \item history of termination criterias,
  \end{itemize}
\item Provide a plot command which allows to graphically see the history of the simplices toward the optimum,
\item Provide query features for the status of the optimization process number of iterations, number of function evaluations, status of execution, function value at initial point, function value at optimal point, etc...
\item Spendley et al. fixed shaped algorithm,
\item Kelley restart based on simplex gradient,
\item O'Neill restart based on factorial search around optimum,
\item Box-like method managing bounds and nonlinear inequality constraints based on arbitrary number of vertices in the simplex.
\end{itemize}


\chapter{Scilab statistical features}

In this chapter, we describe the features which are provided 
in Scilab's core, that is, "out of the box".
Indeed, Scilab provide features such as general statistical 
description of datas, many cumulated density functions 
and can generate uniform and non uniform random variates.
These features are based on several open source libraries, that 
we are analysing in the first section.
A complete overview of these features is provided in the 
second section, where we analyse the full list of functions and 
the numerical methods they use. For the most important 
functions, we provide a sample session where the function is 
used and some plots of the results.

\section{The sources}

In this section, we analyse the libraries which are available 
in Scilab and which provide its statistical features.
The figure \ref{inscilab-libraries} is an overview of the 
libraries which are either Scilab macros or source code, provided
in C, Fortran 77 or as Scilab macros. 

\begin{figure}[htbp]
\begin{tabular}{|l|l|}
\hline
Commands & calerf, erf, erfc, erfcx \\
Routines & CALERF\\
Directory & scilab/modules/elementary\_functions/src/fortran\\
Language & Fortran\\
Download & \url{http://www.kurims.kyoto-u.ac.jp/~ooura/index.html} \\
Author & Takuya Ooura \\
Year & 1996 \\
References & \cite{Algorithm715} \\
\hline
\hline
Name & Labostat \\
Directory & scilab/modules/elementary\_functions/src/fortran\\
Commands & General description functions (center, variance, etc...) \\
Language & Scilab scripts \\
Author & Carlos Klimann \\
Year & 2000 \\
References & \cite{Wonacott1990}, \cite{Saporta2006}\\
\hline
\hline
Name & DCDFLIB \\
Directory & scilab/modules/statistics/src/dcdflib\\
Download & \url{http://www.netlib.org/random/}\\
Commands & Cumulated Density Functions (cdfbet, cdfbin, etc...)  \\
Language & Fortran \\
Author & Barry Brown, W. J. Cody, Alfred H. Morris Jr \\
Year & 1994 for library, 1992 for code by Cody, 1991 for code by Morris \\
References & \cite{abramowitz+stegun1964}, \cite{HartEtAl:1968}, \cite{Algorithm715}, \cite{Kennedy1980}
\cite{Algo708}, \cite{DiDonato1986}\\
\hline
\hline
Name & Randlib \\
Directory & scilab/modules/randlib/src/fortran\\
Download & \url{ftp://odin.mda.uth.tmc.edu/pub/source}\\
& (unavailable at the time of the writing of this report)\\
Commands & grand (for distributions like normal, gamma, chi, etc...)  \\
Language & Fortran \\
Author & Barry Brown, James Lovato, Kathy Russell, John Venier \\
Year & 1997 \\
References & \cite{Ahrens1972}, \cite{358390}, \cite{Devroye86non-uniformrandom},
\cite{AhrensDieter1973}\\
\hline
\end{tabular}
\caption{Statistical libraries available in Scilab}
\label{inscilab-libraries}
\end{figure}


\section{Overview of functions}

The figure \ref{inscilab-fulllist} presents a complete list 
of Scilab statistical functions.

\begin{figure}[htbp]
\begin{tabular}{|l|l|}
\hline
\textbf{Description} & \\
\textbf{of Data} & \\
\hline
center &     cmoment \\
correl &     covar  \\
ftest &     ftuneq  \\
geomean &    harmean  \\
iqr &    labostat  \\
mad &    mean  \\
meanf &    median  \\
moment &    msd  \\
mvvacov &    nancumsum  \\
nand2mean &    nanmax  \\
nanmean &    nanmeanf  \\
nanmedian &    nanmin  \\
nanstdev &    nansum  \\
nfreq &    pca  \\
perctl &    princomp  \\
quart &    regress  \\
sample &    samplef  \\
samwr &    show\_pca  \\
st\_deviation &    stdevf  \\
strange &    tabul  \\
thrownan &    trimmean  \\
variance &    variancef  \\
wcenter & \\
\hline
\end{tabular}
\begin{tabular}{|l|l|}
\hline
\textbf{Special} & \\
\textbf{Functions} & \\
\hline
beta & calerf \\
erf & erfc \\
erfcx & erfinv \\
gamma & gammaln \\
\hline
\hline
\textbf{Random} &\\
\textbf{Number} &\\
\textbf{Generation} &\\
\hline
grand & prbs\_a \\
rand & sprand \\
randpencil &\\
\hline
\hline
\textbf{Cumulated} &\\
\textbf{Density} &\\
\textbf{Functions} &\\
\hline
cdfbet & cdfbin  \\
cdfchi & cdfchn  \\
cdff & cdffnc  \\
cdfgam & cdfnbn  \\
cdfnor & cdfpoi  \\
cdft & \\
\hline
\end{tabular}
\caption{Complete list of statistical features in Scilab}
\label{inscilab-fulllist}
\end{figure}

\subsection{General description functions}

The figure \ref{inscilab-descriptionfunctions} presents the 
general description functions available in Scilab.

\begin{figure}[htbp]
\begin{tabular}{|l|l|}
\hline
Name & Feature\\
\hline
center & center \\
wcenter & center and weight \\
cmoment & central moments of all orders \\
correl & correlation of two variables \\
covar & covariance of two variables \\
ftest & Fischer ratio \\
ftuneq & Fischer ratio for samples of unequal size. \\
geomean & geometric mean \\
harmean & harmonic mean \\
iqr & interquartile range \\
mad & mean absolute deviation \\
mean & mean (row mean, column mean) of vector/matrix entries \\
meanf & weighted mean of a vector or a matrix \\
median & median (row median, column median,...) of vector/matrix/array entries \\
moment & non central moments of all orders \\
msd & mean squared deviation \\
mvvacov & computes variance-covariance matrix \\
nancumsum & cumulative sum of the values of a matrix \\
nand2mean & difference of the means of two independent samples \\
nanmax & max (ignoring Nan's) \\
nanmean & mean (ignoring Nan's) \\
nanmeanf & mean (ignoring Nan's) with a given frequency. \\
nanmedian & median of the values of a numerical vector or matrix \\
nanmin & min (ignoring Nan's) \\
nanstdev & standard deviation (ignoring the Nans). \\
nansum & sum of values ignoring Nan's \\
nfreq & frequence of the values in a vector or matrix \\
pca & computes principal components analysis with standardized variables \\
perctl & computation of percentils \\
princomp & Principal components analysis \\
quart & computation of quartiles \\
regress & regression coefficients of two variables \\
sample & sampling with replacement \\
samplef & sample with replacement from a population and frequences of his values. \\
samwr & sampling without replacement \\
show\_pca & visualization of principal components analysis results \\
st\_deviation & standard deviation (row or column-wise) of vector/matrix entries  \\
stdevf & standard deviation \\
strange & range \\
tabul & frequency of values of a matrix or vector \\
thrownan & eliminates nan values \\
trimmean & trimmed mean of a vector or a matrix \\
variance & variance of the values of a vector or matrix \\
variancef & standard deviation of the values of a vector or matrix \\
\hline
\end{tabular}
\caption{Description of Data functions}
\label{inscilab-descriptionfunctions}
\end{figure}

\subsection{Special functions}

The figure \ref{inscilab-specialfunctions} presents the special functions 
available in Scilab.

\begin{figure}[htbp]
\begin{tabular}{|l|l|}
\hline
Name & Feature\\
\hline
beta & beta function \\
calerf & computes error functions \\
erf & error function \\
erfc & complementary error function \\
erfcx & scaled complementary error function \\
erfinv & inverse of the error function \\
gamma & gamma function \\
gammaln & logarithm of gamma function \\
\hline
\end{tabular}
\caption{Special functions}
\label{inscilab-specialfunctions}
\end{figure}

The figure \ref{inscilab-specialfunctionsdetailed} presents 
a detailed analysis of the location and internal design
of the special functions available in Scilab.

\begin{figure}[htbp]
\begin{tabular}{|l|l|}
\hline
Name & Location / Internals \\
\hline
beta & modules/special\_functions/sci\_gateway/c/sci\_beta.c \\
& switch to dgammacody by W. J. Cody and \\
& L. Stoltz and to betaln from DCDFLIB \\
\hline
calerf & modules/elementary\_functions/src/fortran \\
& by Takuya OOURA \\
\hline
erf & modules/elementary\_functions/macros/erf.sci \\
&  call to calerf \\
\hline
erfc & modules/elementary\_functions/macros/erfc.sci \\
& call to calerf \\
\hline
erfcx & modules/elementary\_functions/macros/erfcx.sci \\
& call to calerf \\
\hline
erfinv & modules/special\_functions/macros/erfinv.sci \\
& rational aproximation of erfinv + 2 Newton's steps\\
\hline
gamma & modules/special\_functions/sci\_gateway/fortran/sci\_f\_gamma.f  \\
& based on dgammacody by W. J. Cody and L. Stoltz \\
\hline
gammaln & modules/elementary\_functions/src/fortran/dlgama.f  \\
& by W. J. Cody and L. Stoltz \\
\hline
\end{tabular}
\caption{Detailed analysis of special functions}
\label{inscilab-specialfunctionsdetailed}
\end{figure}

\subsection{Cumulated density functions}

The figure \ref{inscilab-cdffunctions} presents the cumulated
density functions available in Scilab.

\begin{figure}[htbp]
\begin{tabular}{|l|l|}
\hline
Name & Feature\\
\hline
cdfbet & Beta distribution \\
cdfbin & Binomial distribution \\
cdfchi & chi-square distribution \\
cdfchn & non-central chi-square distribution \\
cdff & F distribution \\
cdffnc & non-central F distribution \\
cdfgam & gamma distribution \\
cdfnbn & negative binomial distribution \\
cdfnor & normal distribution \\
cdfpoi & poisson distribution \\
cdft & Student's T distribution\\
\hline
\end{tabular}
\caption{Cumulated density functions}
\label{inscilab-cdffunctions}
\end{figure}


\subsection{Random number generation}

The figure \ref{inscilab-randomnumbercommands} presents the random
number generators available in Scilab.

\begin{figure}[htbp]
\begin{tabular}{|l|l|}
\hline
Name & Feature\\
\hline
grand & Random number generators \\
prbs\_a & pseudo random binary sequences generation \\
rand & random number generator \\
sprand & sparse random matrix \\
randpencil & random pencil \\
\hline
\end{tabular}
\caption{Random number commands}
\label{inscilab-randomnumbercommands}
\end{figure}

The figure \ref{inscilab-randomnumbercommands} presents a detailed 
analysis of the location and design of the random
number generators available in Scilab.

\begin{figure}[htbp]
\begin{tabular}{|l|l|}
\hline
Name & Location / Internals \\
\hline
grand & modules/randlib/sci\_gateway/c/sci\_grand.c \\
& based on several random number generators\\
prbs\_a & modules/cacsd/macros/prbs\_a.sci \\
& based on rand \\
rand & modules/elementary\_functions/src/fortran/urand.f \\
& by Michael A. Malcolm And Cleve B. Moler \\
sprand & (todo)\\
randpencil & (todo)\\
\hline
\end{tabular}
\caption{Detailed analysis of random number commands}
\label{inscilab-detailrandomnumbercommands}
\end{figure}


\chapter{Statistical Toolboxes}

\url{http://www.scilab.org/contrib/index_contrib.php?page=download&category=DATA%20ANALYSIS%20AND%20STATISTICS}

GLMBOX :generalized statistical linear models analysis. (Dec 2003).
\url{http://www.scilab.org/contrib/index_contrib.php?page=displayContribution&fileID=183}

grocer 1.2 : Comprehensive econometric toolbox
\url{http://www.scilab.org/contrib/index_contrib.php?page=displayContribution&fileID=248}

Hurst : Exponent estimators v2.0
\url{http://www.scilab.org/contrib/index_contrib.php?page=displayContribution&fileID=988}

multilinear regression
\url{http://www.scilab.org/contrib/index_contrib.php?page=displayContribution&fileID=339}

Sci\_R for scilab 5.x
\url{http://www.scilab.org/contrib/index_contrib.php?page=displayContribution&fileID=1138}

stixbox 1.2.5
\url{http://www.scilab.org/contrib/index_contrib.php?page=displayContribution&fileID=184}
statistics toolbox designed for the french examination "agregation de mathematiques"


\chapter{Missing features}

\begin{itemize}
\item Empirical Cumulated Density Function
\item Robust implementation of variance, standard deviation.
See in "Art of Computer Programming" \cite{artcomputerKnuthVol2}, chapter 4.2.2, "Accuracy of Floating
Point Arithmetic", section A or in "Numerical Recipes" \cite{NumericalRecipes}, 
chapter 14.1, "Moments of a Distribution: Mean, Variance, Skewness, and so Forth".
\end{itemize}



\clearpage

%% Bibliography


\addcontentsline{toc}{chapter}{Bibliography}
\bibliographystyle{plain}
\bibliography{statisticsscilab}

\end{document}

