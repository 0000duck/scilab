% Copyright INRIA

\documentclass[11pt]{article}

\textheight=23cm \textwidth=16cm
\topmargin=-1cm
\oddsidemargin=0pt \evensidemargin=0pt \marginparwidth=2cm

\newcommand{\func}[1]{\texttt{#1}}
\newcommand{\T}[1]{\texttt{#1}}

\title{Communication Toolbox}
\author{Claude Gomez}
\date{Manual Version 1.0 for Scilab 2.4}

\begin{document}

\maketitle

\small\textbf{
This is a beta version of the Communication Toolbox. It lacks security
problems when using remote communications (see~\ref{remote}). A good
error trapping is also under development, so you can ``block'' Scilab
when doing mistakes in the names of the linked applications.
}

\bigskip

GeCI is an interactive communication manager created in order to
manage remote executions of programs and allow exchanges of messages
between those programs. It offers the possibility to exploit numerous
machines on a network, as a virtual computer, by creating a
distributed group of independent softwares.

It has been originally developed in the CalICo Project (LaBRI-
Universit\'e Bordeaux I, France) by Nadine Rouillon.

In Scilab, GeCI manages communications between Scilab itself and other
applications (included Scilab itself). In fact, the \T{scilab}
command is a shell script which at last executes the command:

\begin{verbatim}
<Scilab directory>/bin/geci -local <Scilab directory>/bin/scilex
\end{verbatim}

where \verb|<Scilab directory>| is the main directory of Scilab.
We will use  \verb|<Scilab directory>| to denote this directory in
this manual.
This means that Scilab is ready to communicate with others
applications. In particular, this is the way Metanet Windows are
executed in the Metanet toolbox.

In section~\ref{functions}, we explain how to use the Scilab functions of the
Communication Toolbox and in section~\ref{internal} we describe the
way to make an application able to communicate with Scilab.

\section{Functions of the Communication Toolbox}\label{functions}

There are six functions in the Communication Toolbox to make
communications. They are 
\func{CreateLink}, \func{DestroyLink}, \func{ExecAppli},
\func{GetMsg}, \func{SendMsg} and \func{WaitMsg}. They are described
in the on-line manual. We are only going here to describe how to use
them for exchanging messages between two Scilab programs. You will find
in section~\ref{examples} how to communicate between Scilab and other
programs.

After executing Scilab, you can execute another Scilab on the same
computer from the first one and create a link to it and from it by
issuing the commands:
 
\begin{verbatim}
-->h=unix_g("hostname");
-->ExecAppli(SCI+"/bin/scilex",h,"Scilab2")
-->CreateLink("SELF","Scilab2")
-->CreateLink("Scilab2","SELF")
\end{verbatim}

Each application linked to GeCI has a name. Two special names are defined,
\T{"SELF"} always stands for the name of the Scilab program where you
are and \T{"XGeCI"} stands for the first Scilab program.

Now, another Scilab named \T{Scilab2} appears and you can send a
message to it from the first Scilab:

\begin{verbatim}
-->SendMsg("Hi!","How are you?");
\end{verbatim}

And in \T{Scilab2}, you can get it:

\begin{verbatim}
-->[type,msg,apply] = GetMsg()
 apply  =
 
 XGeCI   
 msg  =
 
 How are you?   
 type  =
 
 Hi!   
\end{verbatim}

You do not have to tell to which application you want to send the
message: you send it to all applications linked to you.

You can also send a message from \T{Scilab2} to first Scilab:

\begin{verbatim}
-->SendMsg("Answer","Fine and you?");
\end{verbatim}

and you get it in Scilab:

\begin{verbatim}
-->[type,msg,apply] = GetMsg()
 apply  =
 
 Scilab2   
 msg  =
 
 Fine and you?   
 type  =
 
 Answer   
\end{verbatim}

You can even open a new Scilab, named \T{Scilab3} for instance, link it
to Scilab and/or \T{Scilab2} and exchange messages between them after
having created links. Use the \func{DestroyLink} function to destroy a
link.

Note that there is no notion of client and server. The server is the
GeCI program and all the applications which communicate are at the
same level. That means that the \func{CreateLink}, \func{DestroyLink}
and \func{ExecAppli} functions can be used in any Scilab instance.

\section{Creating applications to communicate with Scilab}\label{internal}

To communicate with Scilab, an application must have been prepared
for, by including a communication library in it. 
The way to do this is described in this section.

Suppose you have a C or fortran program and you want to add in it the
functionalities to communicate using GeCI. For this you need to add
new C functions in the code. For instance, you can create a new C file
which you are going to link with your existing C or fortran
program. You also need to link the program with the following library:

\begin{verbatim}
<Scilab directory>/libs/libcomm.a
\end{verbatim}

At the beginning of the C code, you must include the following header
files:
 
\begin{verbatim}
#include "<Scilab directory>/routines/libcomm/libCalCom.h"
#include "<Scilab directory>/routines/libcomm/libCom.h"
\end{verbatim}

Then you define the messages known by the application, usually:

\begin{verbatim}
static void QuitAppli();
static void EndAppli();
static void ParseMessage();
static void MsgError();

static actions_messages tb_messages[]={
    {ID_GeCI,MSG_QUITTER_APPLI,NBP_QUITTER_APPLI,QuitAppli},
    {ID_GeCI,MSG_FIN_APPLI,NBP_FIN_APPLI,EndAppli},
    {NULL,MSG_DISTRIB_LISTE_ELMNT,NBP_DISTRIB_LISTE_ELMNT,ParseMessage},
    {NULL,NULL,0,MsgError}};
\end{verbatim}

All these functions have one argument of type \T{message}.

\T{QuitAppli} is executed when THIS application terminates.

\T{EndAppli} is executed when an application executed by THIS
application terminates.

\T{MsgError} is executed when there is an error in a message.

\T{ParseMessage} is executed to get the received messages.
For instance, you can write \T{ParseMessage} the following way:

\begin{verbatim}
static char *TheAppli;
static char *TheType;
static char *TheMsg;    

static void ParseMessage(message)
Message message;
{
  int lappli, ltype, lmsg;

  lappli = strlen(message.tableau[0]);
  if ((TheAppli = (char *)malloc((unsigned)sizeof(char)*(lappli + 1)))
      == NULL) {
    return;
  }
  strcpy(TheAppli,message.tableau[0]);

  ltype = strlen(message.tableau[3]);
  if ((TheType = (char *)malloc((unsigned)sizeof(char)*(ltype + 1)))
      == NULL) {
    return;
  }
  strcpy(TheType,message.tableau[3]);

  lmsg = strlen(message.tableau[4]);
  if ((TheMsg = (char *)malloc((unsigned)sizeof(char)*(lmsg + 1)))
      == NULL) {
    return;
  }
  strcpy(TheMsg,message.tableau[4]);
}
\end{verbatim}

and you get:

\begin{itemize}
\item in \T{TheAppli} the name of the application which has sent the message
\item in \T{TheType} the type of the message (an ASCII string)
\item in \T{TheMsg} the message (an ASCII string)
\end{itemize}

Then you must initialize the communications. 

For this, the main function of the 
application must understand the arguments \T{-pipes <pipe1> <pipe2>} which
are automatically given by GeCI when executing it by:

\begin{verbatim}
geci -local <my application>
\end{verbatim}

A simple way to do this is to use the \T{find} function:
\begin{verbatim}
static int find(s,n,t)
char *s;
int n;
char **t;
{
  int i;
  for (i=0; i<n; i++)
    if (!strcmp(s,t[i])) return(i);
  return(-1);
}

int main(argc, argv)
unsigned int argc;
char **argv;
{
  int igeci;
  int p1, p2;
  igeci = find("-pipes",argc,argv);
  if (igeci == -1) exit(1);

  p1 = atoi(argv[igeci+1]); p2 = atoi(argv[igeci+2]);
\end{verbatim}

And you initialize the communications:

\begin{verbatim}
  init_messages(tb_messages,p1,p2);
\end{verbatim}

Then it is possible to use the functions of GeCI. For this, you have
to send messages to GeCI.

\begin{itemize}
\item You can execute an application from your program:

\begin{verbatim} 
  envoyer_message_parametres_var(ID_GeCI,
                                 MSG_LANCER_APPLI,
                                 <appli>,
                                 <host>,
                                 <path appli>,
                                 INS_ID_PIPES,
                                 NULL);
\end{verbatim}

where \T{<appli>} is the name you give to the application you execute,
\T{<host>}
is the name of the host where you want to execute the application and 
\T{<path appli>} is the path of the program of the application on the
host. If you want to execute an application locally on the same host,
you give the name of your host. You can also execute the application
on a remote host on Internet, see \ref{remote}.

Every application has a name. The name of your program has also a name
given automatically. When you have to use it (to link applications for
instance), you can get it by \T{identificateur\_appli()}.

\item To create a (directed) link from the application named \T{<appli1>} to
the application named \T{<appli2>}:

\begin{verbatim} 
  envoyer_message_parametres_var(ID_GeCI,
                                 MSG_CREER_LIAISON, 
                                 <appli1>,
                                 <appli2>,
                                 NULL);
\end{verbatim}

Then you are able to send messages from \T{<appli1>} to \T{<appli2>}.

Note that the two applications must have been executed first by the
preceding message.

\item To destroy a link from \T{<appli1>} to \T{<appli2>}:

\begin{verbatim} 
  envoyer_message_parametres_var(ID_GeCI,
                                 MSG_DETRUIRE_LIAISON, 
                                 <appli1>,
                                 <appli2>,
                                 NULL);
\end{verbatim}

\item To send a message to all linked applications:

\begin{verbatim} 
  envoyer_message_parametres_var(ID_GeCI,
                                 MSG_POSTER_LISTE_ELMNT,
                                 <type>,
                                 <msg>,
                                 NULL);
\end{verbatim}

where \T{<type>} is a string corresponding to the type of the message
and \T{<msg>} is the string corresponding to the message.

Note that before being able to exchange messages, applications must
have been linked.

\item To get a message in an asynchronous way:

\begin{verbatim}
  scanner_messages();
\end{verbatim}
 
\item To wait for a message from \T{<appli>} in a synchronous way:
\begin{verbatim}

  attendre_reponse(<appli>,
                   MSG_DISTRIB_LISTE_ELMNT,
                   NBP_DISTRIB_LISTE_ELMNT);
\end{verbatim}
\end{itemize}

\subsection{Communication between remote hosts}\label{remote}

\emph{\textbf{
CAUTION: With this beta version of the Communication Toolbox, nothing
has been done for adressing possibly security holes and problems when
using the \T{gecid} daemon and remote \T{geci} programs. So use them
very carefully.
}}

\medskip

With GeCI you can also have communications between programs on remote
hosts.

Suppose you are on host \T{h1}, have a local application \T{a1} and
you want to execute an application \T{a2} on host \T{h2} and open a
communication between \T{a1} and \T{a2}.

First, you must have a \T{geci} program on both hosts. Second you must
have a daemon, called \T{gecid}, on the remote host. The C source code
of this daemon is given in the \T{geci} directory of Scilab
distribution.

You have to give the good path of \T{geci} program in the source code
of program \T{gecid} in variable \T{GECI}.

Then, you first start the daemon \T{gecid} on host \T{h2}. 
After, you execute application \T{a1}, Scilab for instance, on host
\T{h1} and use GeCI functions to execute application \T{a2} on host \T{h2}:
you have to give the complete Internet name of host \T{h2} in these
functions.

If application \T{a2} is another Scilab, do not forget to give the
good \T{-display} argument to the \T{scilex} command.

In fact the \T{gecid} daemon will wait for a socket connection on port
2001 and then start \T{geci} on host \T{h2}.

\subsection{Examples}\label{examples}

You will find below two complete C programs as examples.

The first program opens Scilab,
wait for messages and print them. We call the file of this program 
\T{alpha.c}
and we suppose that the main directory of Scilab, 
\verb|<Scilab directory>| is \T{/usr/local/lib/scilab-2.4}.

Compile this program with:
 
\begin{verbatim}
cc -o alpha alpha.c /usr/local/lib/scilab-2.4/libs/libcomm.a
\end{verbatim}

and execute it with (if shell is csh):

\begin{verbatim}
setenv SCI /usr/local/lib/scilab-2.4
/usr/local/lib/scilab-2.4/bin/geci -local alpha
\end{verbatim} 

Then a new Scilab is executed and you can execute in it commands of the form
\T{SendMsg("Hi","How are you?")}. On the console you must see:
\begin{verbatim}
Message received from Scilab
    type: Hi
    message: How are you?
\end{verbatim}

"alpha.c" is given below and in the directory \T{docs/comm} of Scilab
distribution.

\begin{verbatim}
/***************************************************************************/
#include <stdio.h>
#include <string.h>

/* Communications headers */
#include "/usr/local/lib/scilab-2.4/routines/libcomm/libCalCom.h"
#include "/usr/local/lib/scilab-2.4/routines/libcomm/libCom.h"

static void QuitAppli();  
static void EndAppli();
static void ParseMessage();
static void MsgError();

/* Known messages */
static actions_messages tb_messages[]={
  {ID_GeCI,MSG_QUITTER_APPLI,NBP_QUITTER_APPLI,QuitAppli},
  {ID_GeCI,MSG_FIN_APPLI,NBP_FIN_APPLI,EndAppli},
  {NULL,MSG_DISTRIB_LISTE_ELMNT,NBP_DISTRIB_LISTE_ELMNT,ParseMessage},
  {NULL,NULL,0,MsgError}};

static void QuitAppli(message)
     Message message;
{  
  printf("Quit application\n");
  exit(0);
}

static void EndAppli(message) 
     Message message;
{
  printf("End application\n");
}

static void MsgError(message)
     Message message;
{
  printf("Bad received message\n");
}

static char *TheAppli;
static char *TheType;
static char *TheMsg;

/* ParseMessage is executed when a message is received */
static void ParseMessage(message)
     Message message;
{
  int lappli, ltype, lmsg;

  lappli = strlen(message.tableau[0]);
  if ((TheAppli = (char *)malloc((unsigned)sizeof(char)*(lappli + 1)))
      == NULL) {
    return;
  }
  strcpy(TheAppli,message.tableau[0]);

  ltype = strlen(message.tableau[3]);
  if ((TheType = (char *)malloc((unsigned)sizeof(char)*(ltype + 1)))
      == NULL) {
    return;
  }
  strcpy(TheType,message.tableau[3]);

  lmsg = strlen(message.tableau[4]);
  if ((TheMsg = (char *)malloc((unsigned)sizeof(char)*(lmsg + 1)))
      == NULL) {
    return;
  }
  strcpy(TheMsg,message.tableau[4]);
}

static int find(s,n,t)
     char *s;
     int n;
     char **t;
{
  int i;
  for (i=0; i<n; i++)
    if (!strcmp(s,t[i])) return(i);
  return(-1);
}

int main(argc, argv)
     unsigned int argc;
     char **argv;
{
  int igeci;
  int p1, p2;
  char myhost[128];
  /* Scilab application to execute */
  char *scilex = "/usr/local/lib/scilab-2.4/bin/scilex";

  igeci = find("-pipes",argc,argv);
  if (igeci == -1) exit(1);

  p1 = atoi(argv[igeci+1]); p2 = atoi(argv[igeci+2]);

  /* Intialization of communications */
  init_messages(tb_messages,p1,p2);

  /* Get the name of my computer */
  gethostname(myhost,128);

  /* Execute Scilab with name "Scilab" on my local host */
  envoyer_message_parametres_var(ID_GeCI,
                                 MSG_LANCER_APPLI,
                                 "Scilab",
                                 myhost,
                                 scilex,
                                 INS_ID_PIPES,
                                 NULL);

  /* Link THIS application with "Scilab" */
  envoyer_message_parametres_var(ID_GeCI,
                                 MSG_CREER_LIAISON, 
                                 identificateur_appli(),
                                 "Scilab",NULL);
  
  /* Link "Scilab" with THIS application  */
  envoyer_message_parametres_var(ID_GeCI,
                                 MSG_CREER_LIAISON, 
                                 "Scilab",
                                 identificateur_appli(),NULL);

  /* Loop waiting for messages */
  while (1) {
    scanner_messages();
    if (TheType != NULL) {
      printf("Message received from %s\n",TheAppli);
      printf("    type: %s\n",TheType);
      printf("    message: %s\n",TheMsg);
      TheAppli = NULL; TheType = NULL; TheMsg = NULL;
    }
  }
}
/***************************************************************************/
\end{verbatim} 

\bigskip
 
The second program has everything to communicate with another
application linked to GeCI, Scilab for instance.
It waits for messages and print them. We call the file of this program 
\T{beta.c}
and we suppose that the main directory of Scilab, 
\verb|<Scilab directory>| is \T{/usr/local/lib/scilab-2.4}.

Compile this program with:
 
\begin{verbatim}
cc -o beta beta.c /usr/local/lib/scilab-2.4/libs/libcomm.a
\end{verbatim}

Then, being in the directory where \T{beta} lies, execute Scilab and
issue the following Scilab command:
\begin{verbatim}
// get host name of my computer
h=unix_g("hostname")
// execute program beta from Scilab and give it "Beta" as a name
ExecAppli("beta",h,"Beta")
// create a link from Scilab to 
CreateLink("SELF","Beta")
// send a message to "Beta"
SendMsg("Hi","How are you?")
\end{verbatim}

On the console you must seen (written by \T{beta}):
\begin{verbatim}
Message received from Scilab
    type: Hi
    message: How are you?
\end{verbatim}

"beta.c" is given below and in the directory \T{docs/comm} of Scilab
distribution.

\begin{verbatim}
/***************************************************************************/
#include <stdio.h>
#include <string.h>

/* Communications headers */
#include "/usr/local/lib/scilab-2.4/routines/libcomm/libCalCom.h"
#include "/usr/local/lib/scilab-2.4/routines/libcomm/libCom.h"

static void QuitAppli();  
static void EndAppli();
static void ParseMessage();
static void MsgError();

/* Known messages */
static actions_messages tb_messages[]={
  {ID_GeCI,MSG_QUITTER_APPLI,NBP_QUITTER_APPLI,QuitAppli},
  {ID_GeCI,MSG_FIN_APPLI,NBP_FIN_APPLI,EndAppli},
  {NULL,MSG_DISTRIB_LISTE_ELMNT,NBP_DISTRIB_LISTE_ELMNT,ParseMessage},
  {NULL,NULL,0,MsgError}};

static void QuitAppli(message)
     Message message;
{  
  printf("Quit application\n");
  exit(0);
}

static void EndAppli(message) 
     Message message;
{
  printf("End application\n");
}

static void MsgError(message)
     Message message;
{
    printf("Bad received message\n");
}

static char *TheAppli;
static char *TheType;
static char *TheMsg;

/* ParseMessage is executed when a message is received */
static void ParseMessage(message)
     Message message;
{
  int lappli, ltype, lmsg;

  lappli = strlen(message.tableau[0]);
  if ((TheAppli = (char *)malloc((unsigned)sizeof(char)*(lappli + 1)))
      == NULL) {
    return;
  }
  strcpy(TheAppli,message.tableau[0]);

  ltype = strlen(message.tableau[3]);
  if ((TheType = (char *)malloc((unsigned)sizeof(char)*(ltype + 1)))
      == NULL) {
    return;
  }
  strcpy(TheType,message.tableau[3]);

  lmsg = strlen(message.tableau[4]);
  if ((TheMsg = (char *)malloc((unsigned)sizeof(char)*(lmsg + 1)))
      == NULL) {
    return;
  }
  strcpy(TheMsg,message.tableau[4]);
}

static int find(s,n,t)
     char *s;
     int n;
     char **t;
{
  int i;
  for (i=0; i<n; i++)
    if (!strcmp(s,t[i])) return(i);
  return(-1);
}

int main(argc, argv)
     unsigned int argc;
     char **argv;
{
  int igeci;
  int p1, p2;

  igeci = find("-pipes",argc,argv);
  if (igeci == -1) exit(1);

  p1 = atoi(argv[igeci+1]); p2 = atoi(argv[igeci+2]);

  /* Intialization of communications */
  init_messages(tb_messages,p1,p2);

  /* Loop waiting for messages */
  while (1) {
    scanner_messages();
    if (TheType != NULL) {
      printf("Message received from %s\n",TheAppli);
      printf("    type: %s\n",TheType);
      printf("    message: %s\n",TheMsg);
      TheAppli = NULL; TheType = NULL; TheMsg = NULL;
    }
  }
}
/***************************************************************************/
\end{verbatim}
\newpage
\tableofcontents

\end{document}
