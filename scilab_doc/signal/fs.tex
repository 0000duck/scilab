% Copyright INRIA


\def\Figdir{Nfigs_source/}
\def\Diary{Nsource_diary/}
\def\MACROSSIGLIB{../../macros/signal/}
\def\SRCDiary{Nsource_diary}\documentstyle{article}

\begin{document}

\def\verbatok#1{\expandafter \begin{verbatim}
\input{#1}}
\label{fir}
\section{Frequency sampling technique}
\index{frequency sampling!FIR filters}

This technique is based on specification of a set of samples
of the desired frequency response at $N$ uniformly spaced points around the unit circle, where $N$ is the filter length.
The z-transform of an FIR filter is easily shown to be :
%%%%%%%%%%%%%%%%
\begin{eqnarray}
H(z)=\frac{1-z^{-N}}{N}\sum_{k=0}^{N-1}\frac{H(k)}{(1-z^{-1}e^{j(2\pi/N)k})}
\label{e1}
\end{eqnarray}
%%%%%%%%%%%%%%%%
This means that one way of approximating any continuous frequency
response is to {\em sample in frequency}, at $N$ equispaced points around the 
unit circle (the frequency samples), and interpolate between them to obtain
the continuous frequency response. Thus, the approximation error will
be exactly zero at the sampling frequencies and finite between them.
This fact has to be related to the reconstruction of a continuous
function from its samples, as exposed in section~\ref{2.2} for example
for the case of a continuous-time signal.

	The interpolation formula for an FIR filter, that is its frequency response, is obtained by evaluating (\ref{e1}) on the unit circle:
%%%%%%%%%%%%%%%
\begin{eqnarray}
H(e^{j\omega}) &=& \frac{e^{-j\omega(N-1)/2}}{N}\sum_{k=0}^{N-1}\frac{H(k)e^{-jk\pi/N}\sin(N\omega/2)}{\sin(\omega/2-k\pi/N)}\nonumber\\
&=& \frac{e^{-j\omega(N-1)/2}}{N}\sum_{k=0}^{N-1}H(k)S(\omega,k)
\end{eqnarray}
%%%%%%%%%%%%%%%
where
%%%%%%%%%%%%%%%
\begin{eqnarray}
S(\omega,k) &=& e^{-jk\pi/N}\frac{\sin(N\omega/2)}{\sin(\omega/2-k\pi/N)}\nonumber\\
&=&\pm e^{-jk\pi/N}\frac{\sin(N(\omega/2)-k\pi/N)}{\sin(\omega/2-k\pi/N)}
\end{eqnarray}
%%%%%%%%%%%%%%
are the interpolating functions.
Thus, the contribution of every frequency sample to the continuous
frequency response is proportional to \(\sin(N\omega/2)/\sin(\omega/2)\) 
shifted by \(k\pi/N\) in frequency.
	The main drawback of this technique is the lack of flexibility
in specifying the transition band width, which is equal to the number of samples the user decides to put in times \(\pi/N\), and thus is 
strongly related to $N$.
Moreover, the specification of frequency samples in transition bands, giving minimum ripple near the band edges, is not immediate.
Nevertheless, it will be seen, in a later chapter on filter optimization
techniques, that simple linear programming techniques can be used to drastically reduce the error approximation by optimizing only those samples located in the transition bands. To illustrate this point, Figure~\ref{f1} shows the response obtained for a type 1 band pass filter with length 65 : 
first with no sample in the transition bands and second (dashed curve) with one sample 
of magnitude .5 in each of these bands. It is worth noting at this point 
that the linear-FIR design problem with arbitrary frequency response 
specification is more efficiently solved using a minmax approximation 
approach, which is exposed in the next section. \\
%%%%%%%%%%%
\input{\Figdir fstyp12.1.tex}
\dessin{picture Type 1 band pass filter with no sample or one sample in each transition band}{f1}
%%%%%%%%%%
Finally, depending on where the initial frequency sample occurs, two distinct sets of frequency samples can be given, corresponding to the so-called
 type 1 and type 2 FIR filters :\\
 \\
\begin{eqnarray}
f_k&=&\frac{k}{N} \; \; k=0, \ldots ,N-1 \mbox{ for type 1 filters}\nonumber\\
f_k&=&\frac{k+1/2}{N}\; \; k=0,\ldots ,N-1 \mbox{ for type 2 filters}\nonumber
\end{eqnarray}
The type of design is at user's will and depends on the application:
for example, a band edge may be closer to a type 1 than to a type 2 frequency
sampling point. This point is illustrated in Figure~\ref{f2}
for the case of a low pass filter with length 64 and no sample in 
the transition band.
%%%%%%%%%%%
\input{\Figdir fstyp12.2.tex}
\dessin{picture Type 1 and type 2 low pass filter}{f2}
%%%%%%%%%%
The full line (resp. the dashed line) gives the approximated response
for the type 1 (resp. type 2) FIR linear filter.
	We give now the way the two previous examples have been 
generated and the code of the macro {\tt fsfir} which calculates 
the approximated response.
Figure~\ref{f1} was obtained with the following set of instructions :\\
\verbatok{\Diary fstyp12.1.dia}
\end{verbatim}
and Figure~\ref{f2} with :
\verbatok{\Diary fstyp12.2.dia}
\end{verbatim}
Finally, the code of the macro {\tt fsfir} is as follows :
%\verbatok{\MACROSSIGLIB fsfirlin.sci}
%\end{verbatim}
%\verbatok{\MACROSSIGLIB sincd.sci}
%\end{verbatim}
%%%%%%%%%%%%%
\end{document}

